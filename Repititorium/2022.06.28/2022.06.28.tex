\newcommand{\aufgabe}[1]{\item \hfill (#1 Punkte)\\}
\subsection{Probeklausur Aufgaben}
\begin{enumerate}[label=Aufgabe \arabic*:, , leftmargin=*, itemsep=-1ex]
    \aufgabe{10}
    Zeigen Sie mit vollständiger Induktion
    \begin{align*}
        \prod_{k = 0}^{n-1} \cos(2^k x) = \frac{\sin(2^nx)}{2^n \sin(x)}
    \end{align*}
    \aufgabe{4}
    Sei $(a_k)_{k \in \N}$ unbeschränkt und monoton wachsend. Zeige das $a_k$ bestimmt divergent gegen $\infty$ ist.
    \aufgabe{5}
    Ist die Folge konvergent oder divergent?
    \begin{align*}
        a_k := k^{k(\cos(k\pi)-1)}
    \end{align*}
    \aufgabe{6}
    Konvergiert die rekursive Folge? Wenn ja, gebe den Grenzwert an!
    \begin{align*}
        a_0 = 1, a_{n+1} = \frac{a_n}{a_n + 2}
    \end{align*}
    \aufgabe{10}
    Überprüfen Sie beide Reihen auf (absolute) Konvergenz oder Divergenz.
    \begin{align*}
        a) \quad \sum_{k = 1}^\infty \frac{(-1)^k k^2}{e^k}, && b) \quad \sum_{k=1}^\infty (-1)^k \frac{k}{1 + k^2}
    \end{align*}
    \aufgabe{6}
    Ziegen Sie die folgenden Aussagen
    \begin{enumerate}[label=\alph*)]
        \item $2 \leq e \leq 3$
        \item $\underset{x \searrow 0}{\lim} \frac{(\tan(\sqrt{x}))^2}{x} = 1$
    \end{enumerate}
    \aufgabe{6}
    Berechnen Sie die Taylorreihe von $T[f, 0]$
    \begin{align*}
        f: (-\infty, \frac{1}{3}) \rightarrow \R, \quad f(x) := \ln(1 - 3x)
    \end{align*}
    \aufgabe{6}
    Sei $h: [-1, 1] \rightarrow \R$ gegeben durch
    \begin{align*}
        h(x) := \begin{cases}
        -1 & \text{für } x \in [-1, 0),\\
        1 & \text{für } x \in [0,1]
        \end{cases}
    \end{align*}
    Beweise oder widerlege: Es existiert $H:(-1, 1) \rightarrow \R$ mit $H'(x) = h(x)$
    \aufgabe{6}
    Berechnen Sie die Integrale:
    \begin{align*}
        a) \quad \int_2^3 \frac{(\ln x)^2}{x} \ dx && b) \quad \int xe^{-x} dx
    \end{align*}
\end{enumerate}

\subsection{Lösungsideen Alexander Frank}
\newcommand{\additionstheoreme}{\underline{\textbf{Additiontheoreme}}}
\begin{enumerate}[label=Aufgabe \arabic*:, , leftmargin=*]
    \aufgabe{10} %Aufgabe 1
    \begin{enumerate}[label=\arabic*)]
        \item Zeige Induktionsanfang für $n = 0$ (\additionstheoreme)
        \item Die Aussage gilt für ein $n \in \N_0$ ($\N$)
        \item Zeige Induktionsschluss (\additionstheoreme)
        \begin{align*}
            \prod_{k = 0}^n &= \cos(2^n x) \prod_{k=0}^{n-1} \cos(2^k x)\\
            &\overset{I.V}{=} \frac{\cos(2^n x) \sin(2^n x}{2^n \sin(x)}\\
            &= \frac{1}{2} \cos(2^n x) \sin(2^nx) + \frac{1}{2} \cos(2^n x) \sin(2^n x)\\
            &= \frac{1}{2} \sin (2^n x + 2^n x)
        \end{align*}
    \end{enumerate}
    \aufgabe{4} %Aufgabe 2
    \begin{enumerate}[label=\arabic*)]
        \item \begin{align*}
            \forall N > 0\ \exists n \in \N : N < a_n
        \end{align*}
        \item \begin{align*}
            \forall n \in \N: a_{n+1} \geq a_n
        \end{align*}
    \end{enumerate}
    \begin{align*}
        b_k &= \frac{1}{a_k} & \lim_{k \rightarrow \infty} b_k &= 0
    \end{align*}
    \begin{align*}
        \forall \varepsilon > 0 \ \exists n_0 \in \N \ \forall n \geq n_0: |b_k| < \varepsilon
    \end{align*}
    Begründe mit 1) und 2)
    \begin{align*}
        b_k \rightarrow 0 \Rightarrow a_k \rightarrow \pm \infty \overset{2)}{\Rightarrow} a_k \rightarrow + \infty
    \end{align*}
    \aufgabe{5} %Aufgabe 3
    $a_k$ umformen in eine Form
    \begin{align*}
        a_k &= \begin{cases}
        1 & k \text{ gerade}\\
        k^{-2k} = \frac{1}{k^{2k}} & k \text{ ungerade}
        \end{cases}
    \end{align*}
    Berechne beide Grenzwerte $\Rightarrow$ die Folge divergiert
    \aufgabe{6} %Aufgabe 4
    Optional Folge umschreiben 
    \begin{align*}
        a_{n+1} = 1 - \frac{2}{a_n + 2}
    \end{align*}
    \begin{enumerate}[label=\arabic*)]
        \item Beschränktheit nachweisen ($0 < a_n \leq 1$ per Induktion zeigen)
        \begin{align*}
            1, \frac{1}{3}, \frac{1}{7}, \frac{1}{15}, \frac{1}{31}
        \end{align*}
        \item Monotonie $a_{n + 1} < a_n$
        \item Monoton + Beschränkt $\Rightarrow$ Konvergenz
        \item $a = \frac{a}{a +2} \underline{\rightarrow 0}$
    \end{enumerate}
    \aufgabe{10} %Aufgabe 5
    \begin{enumerate}[label=\alph*)]
        \item \textbf{}\\%a)
        Gefühl sagt absolut konvergent\\
        Quotientenkriterium wählen und zeige 
        \begin{align*}
            \frac{(k + 2) ^2 e ^k}{k^2 e^{k+1}} \rightarrow \frac{1}{e} < 1
        \end{align*}
        \item \textbf{}\\%b)
        Gefühl sagt konvergent und nicht absolut konvergent\\
        Leibniz Kriterium anwenden und zeigen
        \begin{align*}
            \frac{k}{k^2 + 1} \rightarrow 0 \\
            \text{monoton}
        \end{align*}
        nicht absolut durch
        \begin{align*}
            \sum_{k = 1}^\infty \frac{k}{k^2 +1} > \sum_{k = 1}^\infty \frac{k}{k^2 + k} = \sum_{k = 1}^\infty \frac{1}{k+1} = \sum_{k = 2}^\infty \frac{1}{k}
        \end{align*}
    \end{enumerate}
    \aufgabe{6} %Aufgabe 6
    \begin{enumerate}[label=\alph*)]
        \item \textbf{}\\%a)
        \begin{align*}
            2 &\leq \sum_{k = 0}^\infty \frac{1}{k!} \leq 3\\
            2 &\leq 1 + 1 + \underbrace{\sum_{k = 2}^\infty \frac{1}{k!}}_{\text{geometrisch größere Reihe}} \leq 1 + 1 + 2 \cdot \sum_{k = 2}^\infty \frac{1}{2^k} = 3
        \end{align*}
        \item \textbf{}\\%b)
        \begin{align*}
            \lim_{x \searrow 0} \frac{\sin^2(\sqrt{x}}{x \cos^2(\sqrt{x}} && \text{L'Hospital 2 mal}
        \end{align*}
    \end{enumerate}
    \aufgabe{6} %Aufgabe 7
    \begin{enumerate}[label=\underline{\arabic*}]
        \item Bilde $f', f'', f''', ...$
        \item Konstruiere $f^{(n)}$
        \item Induktionsbeweis
        \item Taylorreihe 
        \begin{align*}
            \sum_{k = 0}^\infty \frac{f^{(k)} (x_0)}{k!} (x - x_0) ^k
        \end{align*}
        Lösung:
        \begin{align*}
            \sum_{n=1}^\infty \frac{3^n}{n} x^n
        \end{align*}
    \end{enumerate}
    \aufgabe{6} %Aufgabe 8
    Zeichne einmal $h(x)$ auf. Das erinnert dann an $g(x) = (|x|)' \quad x \in \R \backslash \{0\}$
    Zeige dass der Differenzenquotient im Punkt 0 kaputt geht. Und widerlege durch Konstruktion.
    \aufgabe{6} %Aufgabe 9
    \begin{enumerate}[label=\alph*)]
        \item \textbf{}\\%a)
        Substituiere
        \begin{align*}
            \ln x = u\\
            \int_{\ln 2}^{\ln 3} u^2 du
        \end{align*}
        \item \textbf{}\\%b)
        Nutze Partielle Integration
        \begin{align*}
            f(x) &= x & g(x) &= - e^{-x}\\
            f'(x) &= 1 & g'(x) &= e ^{-x}
        \end{align*}
    \end{enumerate}
\end{enumerate}

\subsection{Dinge die auf dem Klausurzettel stehen sollten}
\begin{itemize}
    \item Additiontheoreme zu $\sin$ und $\cos$
\end{itemize}