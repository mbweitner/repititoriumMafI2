\subsection{Aufgabe 1 (Taylor Reihen)}
\begin{align*}
    f(x) &= e^x + e^{-x} & x_0 &= 0
\end{align*}
\begin{enumerate}[label=Schritt \arabic*: ]
    \item Wie weit ist die Funktion differenzierbar?
    \item Bilde die ersten drei Ableitungen $f'(x), f''(x), f'''(x)$ und $f(0), f'(0), f''(0), f'''(0)$.
    \item Bestimme $f^{(k)}(x)$ ($\Rightarrow$ und mache einen Induktionsbeweis)
    \item Berechne $f^{(k)}(x)$
    \item Bilde die Taylorreihe/(polynom von Grad n)
    \begin{align*}
        \sum_{k = 0}^{\infty} \frac{f^{(k)}(x_0)}{k!} (x - x_0)^k
    \end{align*}
    \item Bestimmen Sie den
    \begin{itemize}
        \item Konvergenzradius
        \item \underline{Konvergenzbereich}
    \end{itemize}
    Restglied: $R_n = \frac{f^{(n+1)}(x_0)}{(n+1)!}(x - x_0)^{n+1} < k$
    \tipp{Unterschied Konvergenzradius - Konvergenzbereich}{
        Radius:
        \begin{itemize}
            \item Quotientenkriterium
            \begin{align*}
                \forall x \in A:\exists N \in \N \: \forall n \geq N:\\
                \left|\frac{a_{n+1}}{a_n}\right| < 1\\ 
                r := \underset{n \rightarrow \infty}{\lim} \left|\frac{a_n}{a_{n+1}}\right|
            \end{align*}
            \item Qurzelkriterium
            \begin{align*}
            \forall x \in A: \exists N \in \N \: \forall n \geq N: \sqrt[n]{|a_n|} < 1
        \end{align*}
        \begin{align*}
            r \overset{n \rightarrow \infty}{:=} \frac{1}{\sqrt[0]{|a_n|}} \cmark \text{Limes}
        \end{align*}
        \begin{align*}
            r = \infty \Rightarrow \text{ Konvergenzbereich } A \subseteq \R
        \end{align*}
        \end{itemize}
    }
\end{enumerate}

\subsubsection{Musterlösung Alexander Frank}
\begin{enumerate}[label=Schritt \arabic*: ]
    \item $f \in C (A, \R) \quad A = \R$
    \tipp{In Klausuren meist $C^{\infty}$}{In der Klausur werden eigentlich immer Funktionen gegeben, die unendlich ableitbar sind.}
    \item 
    \begin{align*}
        f'(x) &= e^x - e^{-x}\\
        f''(x) &= e^x + e^{-x}\\
        f'''(x) &= e^x - e^{-x}\\
        f(0) &= 2\\
        f'(0) &= 0\\
        f''(0) &= 2\\
        f''(0) &= 0
    \end{align*}
    \item 
    Aussage:
    \begin{align*}
        f^{(k)}(x) = e^x + (-1)^k e^{-x}
    \end{align*}
    Induktionsanfanng:\\
    \begin{align*}
        f^{(0)}(x) = e^x + e^{-x}\\
        f^{(1)}(x) = e^x - e^{-x}\\
        f^{(2)}(x) = e^x + e^{-x}\\
        f^{(3)}(x) = e^x - e^{-x}\\
    \end{align*}
    Induktionsvoraussetzung:\\
    \begin{center}
        Es existiert ein $k \in \N_0: f^{(k)}(x) = e^x + (-1)^k e^{-x}$
    \end{center}
    Induktionsschritt:\\
    \begin{align*}
        f^{(k+1)}(x) &= (f^{(k)}(x))'\\
        &\overset{I.V.}{=} (e^x + (-1)^k e^{-x})'\\
        &\overset{Kettenregel}{=} e^x + (-1)^k (-1) e^{-x}\\
        &= e^x + (-1)^{k+1} e^{-x}
    \end{align*}
    
    \randnotiz{
    Schätze für jede Aufgabe ($a - d$) ein $f^{(k)}(x)$.
    \begin{align*}
        a_0 &= x        &  c_0 &= -1    & b_0 &= x^2 - x & d_0 &= 1 = 1 \cdot 1\\
        a_1 &= x^2      &  c_1 &= 1     & b_1 &= x^2 - x & d_1 &= 2 = 1 \cdot 2\\
        a_2 &= 2x^3     &  c_2 &= 1     & b_2 &= x^2 - x & d_2 &= 8 = 2 \cdot 2^2\\
        a_3 &= 6x^4     &  c_3 &= -1    & b_3 &= x^2 - x & d_3 &= 48 = 6 \cdot 2^3\\
        a_4 &= 24x^5    &  c_4&= -1      & b_4 &= x^2 - x & d_4 &= 136 = 24 \cdot 2^4\\
        \vdots && \vdots && \vdots && \vdots\\
        a_k &= k! x^{k+1} & c_k &= \sin(k\frac{\pi}{2}) - \cos(k\frac{\pi}{2}) & b_k &= x^{k+2} -(k+1)x & d_k &= 2^k \cdot k!
    \end{align*}
    }
    \item 
    \begin{align*}
        f^{(k)}(x) &= e^x + (-1)^{k} e^{-x}\\
        f^{(k)}(0) &= 1 + (-1)^k = \begin{cases}
        2 & k \text{ gerade}\\
        0 & k \text{ ungerade}\\
        \end{cases}
    \end{align*}
    \item \begin{align*}
        \sum_{k=0}^{\infty} \frac{1 + (-1)^k}{k!} x^k \overset{k = 2i}{=} \sum_{i = 0}^{\infty} \frac{2}{(2i)!} x^{2i}
    \end{align*}
    \item Quotientenkriterium:
    \begin{align*}
        \left| \frac{\frac{2}{2(i +1)!} x^{2(i+1)}}{\frac{2}{(2i)!} x^{2i}}\right| &= \frac{1}{(2i+1)(2i + 2)} |x|^2\\
        &< \overbrace{\frac{1}{(2N + 1)(2N+2)}|x|^2}^{\forall x \in \R \: N := \lceil x \rceil \forall i \geq N} < 1 \quad \square
    \end{align*}
\end{enumerate}



