\subsection{Aufgabe 1 (leichte Version)}
%Roter Kreis um Ausdruck => Es muss $formel$ angegeben werden
\newcommand{\redkleiner}{\color{red}<\color{black}}
\begin{align*}
    a_{n+1} &= \frac{1}{1 + \frac{1}{a_n}} & a_0 &= 1
\end{align*}

\begin{enumerate}[label=\alph*)]
    \item Zeigen Sie, dass $(a_n)_{n \in \N_0}$ beschränkt ist.
    \item Zeige $(a_n)_{n \in \N}$ ist streng monoton
    \item Konvergent?
    \item Geben Sie den Grenzwert an.
\end{enumerate}


\subsubsection{Musterlösung Alexander Frank}
\begin{enumerate}[label=\alph*)]
    \item \textbf{}\\
    Mögliche Beschränkungen, die man zeigen kann:
    \begin{align*}
        0 &\leq a_n \leq 1\\
        0 &\leq a_n \leq x & x > 1\\
        0 &< a_n \leq 1 & \text{optimale Aussage für Aufgabenteil b)}\\
    \end{align*}
    \wissen{
    \begin{align*}
        \frac{\frac{a}{b}}{\frac{c}{d}} = \frac{a \cdot d}{b \cdot c}
    \end{align*}
    }
    Beschränktheit: $\exists a, b \in \R \ \forall n \in \N_0: a \leq a_n \leq b$\\
    Wir wollen zeigen:
    \begin{align*}
        \forall n \in \N_0: 0 \redkleiner\leq a_n \leq 1
    \end{align*}
    Induktionsanfang:
    \begin{align*}
        0 &\redkleiner\leq 1 \leq 1 & \text{wahr} \cmark (n = 0)
    \end{align*}
    Induktionsvoraussetzung:
    \begin{align*}
        \exists n \in \N_0: 0 \redkleiner\leq a_n \leq 1
    \end{align*}
    Induktionsschritt:
    \begin{align*}
        a &\redkleiner\leq a_{n+1} \leq 1 & \overset{\text{Def.}}{\Leftrightarrow}\\
        0 &\redkleiner\leq \frac{1}{1 + \frac{1}{a_n}} \leq 1 & \overset{\text{Brüche}}{\Leftrightarrow}\\
        0 &\redkleiner\leq \frac{1}{\frac{a_n + 1}{a_n}} \leq 1 & \overset{\text{Kehrw. }\cdot}{\Leftrightarrow}\\
        0 &\redkleiner\leq \frac{a_n}{a_n +1} \leq 1 & \overset{\cdot (a_n + 1) \overset{I.V.}{>} 0}{\Leftrightarrow}\\
        0 \cdot (a_n +1) &\redkleiner\leq a_n \leq a_n +1 & \Leftrightarrow\\
        0 &\redkleiner\leq a_n \leq a_n +1
    \end{align*}
    \tipp{Nutzt solange wie möglich Äquivalenzumformungen}{Man sollte so lange wie möglich Äquivalenzumformungen verwenden. Sollten diese irgendwann nicht mehr ausreichen, dann sollte zu Folgerungen übergegangen werden.}
    
    \item \textbf{}\\
    \tipp{Werte einsetzen hilf}{Es ist sehr Hilfreich erst einmal ein paar Werte in die Funktion/Folge/Reihe einzusetzen, um ein Gefühl für diese zu bekommen. Dadurch ist es leichter herauszufinden, was für diese gilt.}
    \randnotiz{Damit man die Aussage mittels Induktion lösen kann muss man eine Aussage wie folgt konstruieren:
    \begin{align*}
        \forall n \in \N_0: &a_n - \frac{1}{1 + \frac{1}{a_n}} > 0\\
        &a_n- \frac{a_n}{a_n +1} > 0\\
        &\frac{a_n(a_n + 1) - a_n}{a_n +1} > 0\\
        &\frac{a_n^2}{a_n + 1}> 0\\
        a_n^2 > 0
    \end{align*}
    \textbf{Nicht aber so:}
    \begin{align*}
        \forall n \in \N_0: a_n - a_{n+1} > 0
    \end{align*}
    Induktionsanfang: $a_0 = 1$ $a_1 = \frac{1}{2}$ $\frac{1}{2}>0$
    }
    \begin{align*}
        \forall n \in \N_0: &a_{n+1} < a_n & \overset{\text{Def.}}{\Leftrightarrow}\\
        \forall n \in \N_0: &\frac{1}{1 + \frac{1}{a_n}} < a_n & \overset{\text{siehe a)}}{\Leftrightarrow}\\
        \forall n \in \N_0: &\frac{a_n}{a_n +1} < a_n & \overset{\cdot (a_n +1) \overset{a)}{>} 0}{\Leftrightarrow}\\
        \forall n \in \N_0: &a_n < a_n^2 + a_n & \overset{- a_n}{\Leftrightarrow}\\
        \forall n \in \N_0: &0 < a_n^2 & \\
    \end{align*}
    Damit $< 0$ gilt muss man bei der Beschränktheit die linke Seite $\leq$ anpassen zu $<$.
    \tipp{Keine Induktion bei Monotonie-Beweis}{Bei der Monotonie reicht es meist durch Umformungen zu zeigen, dass etwas streng monoton ist. Induktionsbeweise sorgen dafür, dass man auf den falschen Weg geleitet wird.}
    \item Die Folge $(a_n)_{n \in \N_0}$ ist streng monoton fallend und beschränkt, daraus folgt $(a_n)_{n \in \N_0}$ ist konvergent.
    \tipp{Vermeide das Cauchy-Kriterium}{Man will das Cauchy-Kriterium nicht verwenden, das ist hässlich und unhandlich. Es wird auch nur für wenige Aussagen gebraucht. In der Klausur wird es keine Aufgabe geben, wo man das Cauchy-Kriterium verwenden muss.}
    \tipp{Rechne zu ende auch wenn du einen Fehler gemacht hast}{Solltest du bemerken, dass du eine Aufgabe nicht ganz richtig hast in einem frühreren Punkt (Hier beispielsweise a)), dann beende deine Aufgabe trotzdem. Du bekommst trotzdem Punkte wenn du das richtige aus deinen falschen Rechnungen schlussfolgerst. Beispiel: Du folgerst in a), dass die Beschränktheit nicht gilt, dann folgere in c) das richtige aus deinem falschen a). Also dann würde die Divergenz folgen.}
    
    \item
    \newcommand{\limunendlich}{\lim_{n \rightarrow \infty}}
    \begin{align*}
        \limunendlich a_n &= a\\
        \limunendlich a_{n+1} &= a
    \end{align*}
    \begin{align*}
        a &= \frac{1}{1 + \frac{1}{a}} & \overset{a, b)}{\Leftrightarrow}\\
        a &= \frac{a}{a + 1} & \Leftrightarrow\\
        a(a +1) &= a & \Leftrightarrow\\
        a^2 + a &= a & \overset{-a}{\Leftrightarrow}\\
        a^2 &= 0 & \Leftrightarrow\\
        a &= 0
    \end{align*}
    $\exists a \in \R: \limunendlich a_n = a$
    \randnotiz{$1 = \frac{\frac{1}{n}}{\frac{1}{n}}$}
    \begin{align*}
        a_{n+1} = \frac{a_n}{a_n +1}
    \end{align*}
    
    \randnotiz{
        Optionen zur Bestimmung auf Konvergenz:
    \begin{enumerate}[label=\arabic*)]
        \item Monotonie \& Beschränktheit
        \item Cauchy-Kriterium (fast nie)
        \item Implizite Darsteluung (sehr selten)
        \begin{itemize}
            \item \begin{align*}
            a_n = \frac{1}{n+1}
        \end{align*}
        \item Beweise mit Induktion
        \end{itemize}
        
    \end{enumerate}
    }

    
\end{enumerate}

\subsection{Tipps für Rekursive Folgen}
\begin{enumerate}[label=\alph*)]
    \item Folgeglieder bestimmen
    \begin{align*}
        a_0, a_1, a_2, a_3, a_4
    \end{align*}
    \item So einfach wie möglich umformen $a_{n+1} = \dots$
    \item Schritte 1-4 für Konvergenz
    \begin{enumerate}[label=\arabic*)]
        \item Zeigen Sie, dass $(a_n)_{n \in \N_0}$ beschränkt ist.
        \item Zeige $(a_n)_{n \in \N}$ ist streng monoton
        \item Konvergent?
        \item Geben Sie den Grenzwert an.
    \end{enumerate}
\end{enumerate}

\begin{enumerate}[label=$\Rightarrow$]
    \item Schranken aus 1) auch später anpassbar
    \item Beschränktheit induktiv zeigen
    \item Monotonie direkt beweisen (Beschränktheit nutzen)
\end{enumerate}

\subsection{Aufgabe 2}
\begin{align*}
    a_{n+1} &= 1 + \frac{1}{1 + a_n} & a_0 = 1
\end{align*}
\begin{enumerate}[label=\arabic*)]
    \item $a_n \in [1, 2]$
    \item $\xmark$
    \item 
    \item $\sqrt{2}$
\end{enumerate}