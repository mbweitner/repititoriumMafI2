\tipp{Für die Klausur Zeitmanagement}{Gehe alle Aufgaben zu beginn einmal durch und schreibe dir eine Bewertung an die einzelnen Aufgaben, für wie schwer man diese hält.\\
Teile halte dir feste Zeiträume für einzelne Aufgaben fest und gehe zur nächsten Aufgabe, wenn die Zeit zum lösen der Aufgabe nicht reicht.
}

\subsection{Aufgabe 1}
Beweisen Sie:
Für alle $n \in \N$ gilt:
\begin{align*}
    n^3 + 5n \text{ ist durch 6 teilbar}
\end{align*}

%\subsubsection{Versuch die Aufgabe selber zu lösen}
%I.A: ($n = 1$)
%\begin{align*}
%    1^3 + 5 \cdot 1 = 1 + 5 = 6 \Rightarrow 6 \% 6 = 0 \checkmark
%\end{align*}
%IV:
%\begin{center}
%    Für ein beliebiges aber festes $n \in \N$ gilt die Aussage.
%\end{center}
%\textbf{}\\
%IS: ($n \rightarrow n+1$)
%\begin{align*}
%    (n+1)^3 + 5(n+1) &= n^3 + 3n^2 + 3n + 1 + 5(n+1)\\
%    &=n^3 + 3n^2 + 3n + 1 + 5n + 5\\
%\end{align*}


\subsubsection{Musterlösung Alexander Frank}
\tipp{Aussagen greifbar machen}{Vereinfache die Aussage, sodass sie einfacher zu beweisen ist.}
\begin{align*}
    A(n) &\Leftrightarrow 6 | (n^3 + 5n)\\
    &\Leftrightarrow \frac{1}{6}(n^3 + 5n) \in \N
\end{align*}
\textbf{}\\
Induktionsanfang: ($A(1)$)
\begin{align*}
    6 &| (1^3 + 5)\\
    6 &| 6\\
    &\hspace{2em}\checkmark
\end{align*}
\textbf{}\\
Induktionsvoraussetzung:
\begin{center}
    Es gelte für ein $n \in \N$ die Aussgae $A(n)$.
\end{center}
\tipp{Induktionsvoraussetzung für \underline{\textbf{ein}} $n$}{Es darf nicht geschrieben werden, dass die Aussage für alle $n \in \N$ gilt.\\
Eine andere Formulierung wäre:
\begin{center}
    Es existiert ein $n \in \N$ für alle $k \leq n$ gilt $A(k)$.
\end{center}
}

Induktionsschritt: ($n \rightarrow n+1$)
\tipp{Aussage mit $n+1$ aufschreiben }{Schreibe immer die Aussage die zu zeigen ist auf.\\
Dies hilft, das Ziel besser im Auge zu behalten und den Weg nicht zu verlieren.
}
\begin{align*}
    A(n+1) \Leftrightarrow 6 | \big((n+1)^3 + 5(n+1)\big)
\end{align*}
\tipp{Wähle immer die schwerere Seite zuerst}{
Es ist leichter von der von der schwereren Seite zur Leichten zu gelangen.\\
Analogie: Vom Berg runter ist leichter als den Berg herauf zu gelangen.
}
\tipp{Induktionsvoraussetzung benutzen}{Irgendwann \underline{muss} immer die Induktionsvoraussetzung eingesetzt werden.}
\begin{align*}
    6 | \big((n+1)^3 + 5(n+1)\big) &\Leftrightarrow 6 | \big((n^3 + 3n^2 + 3n +1) + 5n +5\big)\\
    &\Leftrightarrow 6 |\big(\underbrace{(n^3 +5n)}_{\text{Ist durch 6 teilbar}} + (3n^2 +3n +6)\big)\\
    &\overset{\text{I.V.}}{\Leftrightarrow} 6 | (3n^2 + 3n + 6) \\
    &\Leftrightarrow 6| \big(3n(n+1) + \underbrace{6}_{\text{Ist durch 6 Teilbar}}\big)\\
    &\Leftrightarrow 6|\big(3n(n+1)\big)\\
    &\Leftrightarrow 6|3 \cdot \sum_{k=1}^{n} 2k\\
    &\Leftrightarrow 6|6 \cdot \sum_{k=1}^{n}k
\end{align*}

\tipp{Induktionsbeweis in Induktionsbeweis}{Fällt einem die Formel $\sum_{k=1}^{n} 2k = n(n+1)$ kann man auch einfach einen Induktionsbeweis im Induktionsbeweis führen wie folgt:
\begin{align*}
    B(n) \Leftrightarrow 6 | 3(n+1)n
\end{align*}
I.A.:
\begin{align*}
    B(1) \Leftrightarrow 6 | 6 \checkmark
\end{align*}
I.V.:
\begin{center}
    Die Aussage $B(n)$ gilt für ein beliebiges aber festes $n \in \N$
\end{center}
I.S:
\begin{align*}
    B(n+1) &\Leftrightarrow 6 | 3(n+2)(n+1)\\
    &\Leftrightarrow 6 | \big(3n(n+1) + 6(n+1)\big)\\
    &\overset{I.V.}{\Leftrightarrow} 6 | 6(n+1)
\end{align*}
}

\spickzettel{$\sum_{k=1}^{n} 2k = n(n+1)$}

\newpage
\subsection{Aufgabe 2}

Beweisen Sie, dass für alle $n \in \N$ die Aussage
\begin{align*}
    (1 + x)^n \leq 1 + (2^n -1)x
\end{align*}
für alle $x \in [0, 1]$ gilt.

\subsubsection{Musterlösung Alexander Frank}

\begin{align*}
    A(n) \Leftrightarrow (1 + x)^n \leq 1 + (2^n -1)x &&\forall x \in [0, 1]
\end{align*}

Induktionsanfang:
\begin{align*}
    A(1) &\Leftrightarrow (1 + x)^1 \leq 1 +(2^1 - 1)x &&\forall x \in [0, 1]\\
    &\Leftrightarrow (1 +x) \leq 1 + x  &&\forall x \in [0, 1]
\end{align*}

Induktionsvoraussetzung:
\begin{center}
    Es gelte $A(n)$ für $n \in \N$
\end{center}

Induktionsschritt:
\begin{align*}
    A(n + 1) \Leftrightarrow (1 + x)^{n+1} \leq 1 + (2^{n+1} - 1)x
\end{align*}
\tipp{Ausrufezeichnen drüber setzen}{Wenn man ein ! drüber setzt, dann heißt das, man möchte, dass die Aussage gilt, allerdings gibt es noch kein Beweis dazu.}
\begin{align*}
    1 + (2^{n+1} -1)x &\Leftrightarrow 1 + (2^n + 2^n -1)x\\
    &\Leftrightarrow 1 + (2^n -1)x + 2^nx\\
    &\overset{I.V.}{\geq} (1 + x)^n + 2^nx \overset{!}{\geq} (1 + x)^{n+1}\\
    &\Leftrightarrow (1 + x)^n + 2^nx \overset{!}{\geq} (1 + x) (1+x)^n\\
    &\Leftrightarrow (1 + x)^n + 2^n \overset{!}{\geq} (1 + x)^n + x(^+x)^n\\
    &\Leftrightarrow 2^nx \overset{!}{\geq} (1 +x)^nx
\end{align*}
1. Fall ($x = 0$)
\begin{align*}
    0 \geq 0
\end{align*}
2. Fall ($x > 0$)
\begin{align*}
    2^n \overset{!}{\geq} (1 + x)^n
\end{align*}
\begin{align*}
    &\Leftrightarrow 2^n \geq (1 + 1)^n \geq (1 + x)^n
\end{align*}