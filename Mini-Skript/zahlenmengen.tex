\subsection*{Körper}
%%%%%%%%%%%%%%%%%%%%%%%%%%%%%%%%Hier nochmal drüber schauen viellecht passt es ja %%%%%%%%%%%%%%%%%%%%%%%%%%%%%
Hier könnten vielleicht noch die Körperaxiome und die Folgerungen daraus hin. 
%%%%%%%%%%%%%%%%%%%%%%%%%%%%%%%%Hier nochmal drüber schauen viellecht passt es ja %%%%%%%%%%%%%%%%%%%%%%%%%%%%%
\subsection*{Bruchrechenregeln}
\begin{enumerate}[label=\alph*., noitemsep]
    \item $\frac{a}{b} = \frac{c}{d} \eq ad = bc$
    \item $\frac{ae}{be} = \frac{a}{b}$
    \item $\frac{a}{b} \pm \frac{c}{d} = \frac{ad \pm bc}{bd}$
    \item $\frac{\frac{a}{b}}{\frac{e}{d}} = \frac{ad}{be}$
\end{enumerate}
\subsection*{Folgerungen für Ungleichungen}
In einem geordneten Körper $K$ gilt für beliebige Elemente $a, b, c, d \in K$ und $x \in K\backslash \{0\}$
\begin{enumerate}[label=\alph*., noitemsep]
    \item $(a < b) \lor (a > b) \lor (a = b)$
    \item $(a < b) \land (b < c) \Ra a < c$
    \item $(a < b) \land (c \leq d) \Ra a + c < b + d$
    \item $(a < b) \land (x > 0) \Ra ax < bx$\\
          $(a < b) \land (x < 0) \Ra ax > bx$
    \item $a < b \eq a > - b$
    \item $x^2 := x \cdot x > 0$
    \item $0 < a < b \eq 0 < b^{-1} < a^{-1}$
\end{enumerate}
\subsection*{Betrag und Folgerungen}
\begin{align*}
    |x| := \begin{cases}
    x &\text{, falls } x \geq 0\\
    -x &\text{, falls } x < 0
    \end{cases}
\end{align*}
Es gelten folgende Regeln:
\begin{enumerate}[label=\alph*., noitemsep]
    \item $|x| \geq 0 \land (|x| = \eq x = 0)$
    \item $|x \cdot y| = |x| \cdot |y|$
    \item $(|x| < \varepsilon) \eq (x < \varepsilon) \land (-\varepsilon < x) \eq (-\varepsilon < x < \varepsilon)$\\
          $(|x| \leq \varepsilon) \eq (x \leq \varepsilon) \land (-\varepsilon \leq x) \eq (-\varepsilon \leq x \leq \varepsilon)$
    \item $|x + y| \leq |x| + |y| (\text{Dreiecksung.})$
    \item $||x| - |y|| \leq |x - y| (\text{umgekehrte.D.})$
\end{enumerate}
\subsection*{Metrik}
Sei $A$ eine Menge. Wir nennen eine Abbildung $d:A \times A \ra \R$ eine Metrik auf $A$, wenn für alle $x, y, z \in A$ drei Eigenschaften erfüllt sind:
\begin{enumerate}[label=\alph*., noitemsep]
    \item Positive Definitheit
    \begin{align*}
        d(x, y) > 0 \text{ für } x \neq y\\
        d(x, y) = 0 \text{ für } x = y
    \end{align*}
    \item Symmetrie $d(x, y) = d(y, x)$
    \item Dreiecksungleichung
    \begin{align*}
        d(x, y) \leq d(x, z) + d(z, y)
    \end{align*}
\end{enumerate}
Beispielmetrik: $d:\R \times \R \ra \R$ mit $d(x, y) = |x - y|$
\subsection*{Gaußklammern}
Sei $x \in \R$ und $m, n \in \Z$:
\begin{enumerate}[label=\alph*), noitemsep]
    \item $m \leq x < m+1$
    \item $n - 1 < x \leq n$
\end{enumerate}
$\lfloor x \rfloor := m\quad\quad$ $\lceil x \rceil := n$
\subsection*{Modulo}
Sei $m, r, z \in \Z$ und $n \in \N$:
$z = mn + r$\\
$r = z \mod n$
\subsection*{Komplexe Zahlen}

\subsubsection*{Definition}
$\C := \{(a, b) \in \R \times \R\}$
Für $z = (a, b)$ ist $Re(z) := a$ der Realteil und $Im(z) := b$ der Imaginärteil
\subsubsection*{Rechnen mit \texorpdfstring{$\C$}{Komplexen Zahlen}}
$z_1 + z_2 := (a_1 + a_2, b_1 + b_2)$\\
$z_1 \cdot z_2 := (a_1a_2 - b_1b_2, a_1b_2 + a_2b_1)$\\
Nullelement $(0,0)$\\
Einselement $(1, 0)$\\
$-(a, b) = (-a, -b)$(Negativelement)\\
$(a, b)^{-1} = \left(\frac{a}{a^2 + b^2}, \frac{-b}{a^2 + b^2}\right)$ (Inverses Element)
\subsubsection*{Konjugation und Betrag}
$\bar z := (a, -b) = a - ib$\\
$|z| := \sqrt{z \bar z} = \sqrt{a^2 + b^2}$
\subsubsection*{Konjugationseigenschaften und Metrik}
$d(z_1, z_2) = |z_1 - z_2|$\\
$\overline{z_1 \cdot z_2} = \bar z_1 \dot \bar z_2 \quad\quad$
$\overline{z_1 + z_2} = \bar z_1 + \bar z_2$
\subsection*{Summen, Produkte, ...}
\subsubsection*{Def. Summe, Produkt}
\begin{align*}
    &\sum_{k = m}^n a_k := \begin{cases}
    a_m + a_{m+1} +  \\
    ... + a_n & m \leq n\\
    0 &\text{,sonst}
    \end{cases}\\
    &\prod_{k = m}^{n} a_k :=\begin{cases}
    a_m \cdot a_{m+1} \cdot  \\
    ... \cdot a_n & m \leq n\\
    1 &\text{, sonst}
    \end{cases}
\end{align*}
\subsubsection*{Potenzen}
$x^n := \prod\limits_{k = 1}^n x$\\
Für $x \neq 0$: 
$x^{-n} := \frac{1}{x^n} \quad\quad x^0 := 1$
\subsubsection*{Rechenregeln}
$a^n a^m = a^{n + m}\quad$
$(a^n)^m = a^{nm}\quad$
$a^n b^n = (a \cdot b)^n$
\subsubsection*{Fakultät}
$n! := \prod\limits_{k=1}^n k = 1 \cdot 2 \cdot ... \cdot n \quad\quad 0! = 1$
\subsubsection*{Binomialkoeffizient}
\begin{align*}
    &\binom{n}{k} := \begin{cases}
    \frac{n!}{(n - k)! \cdot k!} = \\
    \frac{n \cdot (n-1)\cdot ... \cdot (n-k +1)}{k \cdot (k-1) \cdot ... \cdot 1} & n\geq k\\
    0 & n < k
    \end{cases}
\end{align*}
$\forall n, k \in \N_0: \binom{n}{k} + \binom{n}{k+1} = \binom{n +1}{k+1}$
\subsubsection*{Binomischer Lehrsatz}
Für $a, b \in \R$ und $n \in \N$:
$(a + b)^n = \sum\limits_{k = 0}^n \binom{n}{k} a^{n - k}b^k$
\subsubsection*{Bernoullische Ungleichung}
$\forall x \in \R, x \geq -1, n \in \N_0$ gilt:
$(1 + x)^n \geq 1 + nx$
\subsubsection*{Satz 2.49}
$\forall x \in \R$ mit $x \geq 0$ und $\forall n \in N$ mit $n \geq 2$ gilt
$(1 + x)^n \geq \frac{n^2 x^2}{4}$