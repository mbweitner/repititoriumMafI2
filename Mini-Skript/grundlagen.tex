\subsection*{Aussagenlogik}

\subsubsection*{äquivalente Aussagen}
\begin{adjustwidth}{-0.3cm}{0pt}
\begin{align*}
    \begin{array}{l}
        \neg(\neg A) \eq A \text{ dop. Negation} \\
        A \land (A \lor B)  \text{ Absorpition} \\
        \text{Kommutativität} \\
        A \land B \eq B \land A \\ 
        A \lor B \eq B \lor A \\ \text{} \\
        \text{de Morgan} \\
        \neg (A \land B)    \eq \neg A \lor \neg B \\ 
        \neg (A \lor B)     \eq \neg A \land \neq B \\ \text{} \\
        \text{Assoziativität} \\
        A \lor (B \lor C)   \eq (A \lor B) \lor C \\ 
        A \lor (B \land C)  \eq (A \lor B) \land (A \lor C) \\ \text{Distributivität} \\
        A \land (B \land C) \eq (A \land B) \land C \\ \text{} \\
        A \land (B \lor C)  \eq (A \land B) \lor (A \land C) \\ \text{} \\
    \end{array}
\end{align*}
\end{adjustwidth}

\subsubsection*{All- und Existenzquantor}
\underline{All-Quantor}: $\forall$\\
\glqq Für alle $n$ aus $M$ gilt: $A(n)$\grqq\\
$\forall n \in M: A(n)$\\
\underline{Existenz-Quantor}:\\
\glqq Es existiert mindestens ein $n$ aus $M$, für das gilt: $A(n)$\grqq\\
$\exists n \in M: A(n)$
\subsection*{Mengen}

\subsubsection*{Mengenverknüpfungen}
\begin{align*}
    \begin{array}{l}
        \text{Vereinigung}\\
        A\cup B := \{m | m \in A \lor m \in B\}\\
        \text{Schnitt}\\
        A \cap B := \{m | m \in A \land m \in B\}\\
        \text{Differenz}\\
        A \backslash B := \{m | m \in A \land m \notin B\}\\
        \text{Kartesisches Produkt}\\
        A \times B := \{(m, n) | m \in A \land n \in B\}\\
        \text{Verallgemeinerung Vereinigung}\\
        \bigcup_{M \in \mathcal{N}} M := \{m | \exists M \in \mathcal{N}: m \in M\}\\
        \text{Verallgemeinerung Schnitt}\\
        \bigcap_{M \in \mathcal{N}} M := \{m | \forall M \in \mathcal{N}: m \in M\}\\
        \text{Komplement}
        A^c = U \backslash A\\
    \end{array}
\end{align*}
\subsection*{Beweistechniken}
\subsubsection*{Direkter Beweis}
Folgerungen Umformungen von bereits bewiesenen Aussagen.\\
Bei Äquivalenzen müssen beide Richtungen gezeigt werden:\\
$(A \eq B) \eq ((A \Ra B) \land (B \Ra A))$
\subsubsection*{Indirekter Beweis - Kontraposition}
Beweis erfolgt indem dei Kontraposition gezeigt wird (rechte Seite)\\
$(A \Ra B) \eq (\neq B \Ra \neq B)$\\
\subsubsection*{Widerspruchbeweis}
Zeige, dass die gegenteilige Aussage zu einer Falschen Aussage führt. Ist dies der Fall, so gilt A.\\
$((\neg A \Ra C) \land \neq C) \Ra A$
\subsubsection*{Vollständige Induktion}
\begin{enumerate}[noitemsep]
    \item IA: Beweise $A(n = 1)$
    \item IV: Für ein beliebiges aber festes $n \in \N$ gilt $A(n)$
    \item IS: Beweise $A(n) \Ra A(n+1)$
\end{enumerate}

