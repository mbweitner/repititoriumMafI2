\subsection*{Def. Reihe konvergent}
%$\sum_{k = 1}^\infty a_k = a_1 + ...$ eine Reihe. $s_n = \sum_{k = 1}^{n} a_k$ die n-te Teilsumme der Reihe. 
Folge der Teilsummen konvergent $\Ra$ Reihe konvergent. Sonst divergent.
\subsection*{Cauchy-Konvergenzkrit.}
Reihe konvergiert g.d.w. gilt: $\forall \varepsilon > 0 \exists n_0\in \N \forall n \geq m \geq n_0: \left|\sum\limits_{k = m}^{n} < \varepsilon \right|$ 
\subsection*{Notw. Konvergenzkrit.}
$\sum_{k=1}^\infty a_k$ konvergente Reihe $\Ra$ Folge $(a_k)$ ist Nullfolge $\Ra \lim_{k\ra \infty} a_k = 0$.
\subsection*{Teilsummenbeschränktheit}
$\sum_{k=1}^\infty a_k$ mit $a_k \geq 0$ $\forall k \in \N$ konvergiert g.d.w. Folge der Teilsummen beschränkt.
%%%\subsection*{Rechenregeln konv. Reihen}
%$\sum_{k=1}^\infty a_k$, $\sum_{k = 1}^\infty b_k$ konverg. Reihen:
%%%\begin{enumerate}[label=\alph*., noitemsep]
%%%    \item $\sum_{k = 1}^\infty (a_k \pm b_k)$ konvergent. Für die Grenzwerte gilt:$\sum_{k = 1}^\infty (a_k \pm b_k) = \sum_{k = 1}^\infty a_k \pm \sum_{k = 1}^\infty b_k$
%%%    \item $\sum_{k = 1}^\infty c \cdot a_k$ konvergent für $c \in \R$. Es gilt:$\sum_{k = 1}^\infty c \cdot a_k = \cdot \sum_{k = 1}^\infty a_k$
%%%    \item $\forall l \in N \ l > 0: \sum_{k = l}^\infty a_k$ konvergiert $\eq \sum_{k = 1}^\infty a_k$ konvergiert
%%%    \item Gilt $a_k \leq b_k \forall k \in \N: \sum_{k = 1}^\infty a_k \leq \sum_{k = 1}^\infty b_k$
%%%\end{enumerate}
\subsection*{Def. absolute Konvergenz}
$\sum_{k = 1}^\infty a_k$ abs. k. $\eq \sum_{k = 1}^\infty |a_k|$ konv.
\subsubsection*{Reihenumordnung}
$\sum_{k = 1}^\infty a_k$ abs. konv. $\Ra$ Jede Umordnung konverg. gegen selben Grenzw.
\subsubsection*{abs. Konv. \texorpdfstring{$\Ra$}{folgt} Konvergenz}
$\sum_{k = 1}^\infty a_k$ abs. konv. $\Ra$ konvergent
\subsection*{Cauchy-Produkt}
$\sum_{k = 0}^\infty a_k, \sum_{k = 0}^\infty b_k$ abs. konverg..Für $n \in \N$ sei $c_n := \sum_{k = 0}^n a_k \cdot b_{n - k}$, dann ist $\sum_{k = 0}^\infty = \left(\sum_{k = 0}^\infty a_k\right) \cdot \left(\sum_{k = 0}^\infty b_k\right)$ abs. konv.
\subsection*{Konvergenzkriterien}

\subsubsection*{Leibnitz-Kriterium}
$(a_k)$ monoton fallende Folge mit $\forall k \in \N a_k \geq 0$ mit $\lim_{k \ra \infty}a_k = 0$, dann konvergiert $\sum_{k = 1}^\infty (-1)^k a_k$.
\subsubsection*{Majorantenkriterium}
$\sum_{k = 1}^\infty c_k$ konvergent mit $\forall k \in \N: c_k \geq 0$. Wenn für $\sum_{k = 1}^\infty a_k \exists k_0 \in \N$, sodass $\forall k \geq k_0$ gilt $|a_k| \leq c_k$, dann konvergiert $\sum_{k = 1}^\infty a_k$ absolut.
\subsubsection*{Minorantenkriterium}
$\sum_{k = 1}^\infty c_k$ divergent mit $\forall k \in \N: c_k \geq 0$. Wenn für $\sum_{k = 1}^\infty a_k \exists k_0 \in \N$, sodass $\forall k \geq k_0$  gilt $a_k \geq c_k$, dann divergiert $\sum_{k = 1}^\infty a_k$
\subsubsection*{Wurzelkriterium}
%hier könnte nur die Limesform ausreichen %entfernen
%\begin{enumerate}
%    \item Wenn festes $q \in \R$ mit $0 < q < 1$ und $k_0 \in \N$ existiert, sodass $\forall k \geq k_0: \sqrt[k]{|a_k|} \leq q$, dann konvergiert $\sum_{k = 1}^\infty a_k$ absolut
%    \item $\exists k_0 \in \N$, sodass $\forall k \geq k_0: \sqrt[k]{|a_k|} \geq 1$, dann divergiert $\sum_{k = 1}^\infty a_k$
%\end{enumerate}
%Limesform:\\
$\exists a = \lim_{k \ra \infty} \sqrt[k]{|a_k|}$, dann gilt:
%\begin{itemize}[leftmargin = *, noitemsep]
     $\bullet a < 1 \Ra$ absolut konvergent\\
     $\bullet a > 1 \Ra$ divergent\\
     $\bullet a = 1 \Ra$ unwissend
%\end{itemize}
\subsubsection*{Quotientenkriterium}
%\begin{enumerate}[label=\alph*., noitemsep]
%    \item Wenn festes $q \in \R$ mit $0 < q < 1$ und $k_0 \in \N$ existiert, sodass $\forall k \geq k_0: a_k \neq 0 \land \left| \frac{a_{k+1}}{a_k} \right| \leq q$, dann konvergiert $\sum_{k = 1}^\infty a_k$ absolut.
%    \item $\exists k_0 \in \N$, sodass $\forall k \geq k_0: a_k \neq 0 \land \left| \frac{a_{k+1}}{a_k} \right| \geq 1$, dann divergiert $\sum_{k = 1}^\infty a_k$
%\end{enumerate}
%Limesform:\\
$\exists a = \lim_{k \ra \infty} \left| \frac{a_{k+1}}{a_k} \right|$, dann gilt:
%\begin{itemize}[leftmargin=*, noitemsep]
    $\bullet a < 1 \Ra$ konvergiert absolut\\
    $\bullet a > 1 \Ra$ divergiert\\
    $\bullet a = 1 \Ra $ unwissend 
%\end{itemize}
\subsection*{Potenzreihen}
\subsubsection*{Definition}
%Folge $(a_k)$, $x, x_0 \in \R$, dann Potenzreihe $P(x, x_0)$ mit Entwicklungspunkt $x_0$ definiert als: 
$P(x, x_0) = \sum_{k = 0}^\infty a_k \cdot (x - x_0)^k$\\
$x_0 = 0$, $P(x, 0) = \sum_{k = 0}^\infty a_k \cdot x^k$
\subsubsection*{Konvergenz von Potenzr.}
\begin{enumerate}[label=\alph*., noitemsep]
    \item $P(x, x_0)$ konverg. in $c \Ra$ konverg. absolut $\forall x: |x - x_0| < |c - x_0|$
    \item Konverg. $P(x, x_0)$ in $c$ nicht abs. $\Ra$ divergiert $P(x, x_0) \forall |x - x_0| > |c - x_0|$
\end{enumerate}
\subsubsection*{Def. Konvergenzradius}
Sei $P(x, x_0)$ Potenzreihe. $\exists r \in \R_{\geq 0}$, dass $P(x, x_0)$ $\forall |x- x_0| < r$ konvergiert und $\forall |x - x_0| > r$ divergiert, dann ist $r$ der Konvergenzradius.
\subsubsection*{Konvergenzr. bestimmen}
$\bullet \lim_{n \ra \infty} \sqrt[n]{|a_n|} < 1$\\
$\bullet \lim_{n \ra \infty} |\frac{a_{n+1}}{a_n}| < 1$\\
Umformen:\\
$\bullet r = \lim_{n \ra \infty} \frac{1}{\sqrt[n]{|a_n|}}$
$\bullet r = \frac{|a_n|}{|a_{n+1}|}$
\subsection*{Exponentialreihe}
\subsubsection*{Definition}
$\exp(x) = \sum_{k = 0}^\infty \frac{x^k}{k!}$. $e := \exp(1)$ gilt
\subsubsection*{Konvergenz von Exp.Reihen.}
$\forall x \in \R: \exp(x)$ absolut konvergent.
\subsubsection*{Eigenschaften}
%\begin{enumerate}[label=\alph*., noitemsep]
    a) $\forall x, y \in \R: \exp(x + y) = \exp(x) \cdot \exp(y)$
    b) $\forall x \in \R: \exp(-x) = \frac{1}{\exp (x)}$
    c) $\forall x \in \R: \exp(x) > 0$
    d) $\forall n \in \Z: \exp(n) = e^{n}$ 
%\end{enumerate}
\subsection*{\texorpdfstring{$\exp$}{Exponentialfunktion} als Folgengrenzwert}
Gilt $\forall x \in \R$: $\exp(x) = \sum_{k = 0}^\infty \frac{x^k}{k!} = \lim\limits_{n \ra \infty} \left(1 + \frac{x}{n}\right)^n$. Für $x = 1$ gilt: $e = \sum_{k = 0}^\infty \frac{1}{k!} = \lim\limits_{n \ra \infty}\left(1 + \frac{1}{n}\right)^n$

