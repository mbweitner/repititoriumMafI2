%\subsection*{Definition}
%$A, B$ nichtleere Mengen.Funktion $f$ ordnet jedem $x \in A$ eindeutig $y \in B$ zu. Schrift:$A \ra B$. Zugeordnetes Element auch als $f(x)$.\\
%$f:A \ra B$,
%$A$ Definitionsbereich,\\
%$B$ Bild-/Zielbereich\\
%$f(A) \subseteq B$ Bildmenge/Bild von $f$
\subsection*{Injektiv, ...}
%\begin{enumerate}[label={}, noitemsep, leftmargin=*]
 Injektiv: $x_1 \neq x_2 \Ra f(x_1) \neq f(x_2)$\\
 Surjektiv:$\forall y \in B \exists x \in A: f(x) = y$\\
 Bijektiv: Injektiv + Surjektiv
%\end{enumerate}

%%%\subsection*{Rechenregeln}
%Sei $f, g: A \ra \R$ Funktionen und $c \in \R$. Dann gilt:
%\begin{itemize}[noitemsep, leftmargin=*]
%%%     $\bullet (f + g)(x) := f(x) + g(x)$\\
%%%     $\bullet (cf)(x) := c f(x)$\\
%%%     $\bullet (f \cdot g)(x) := f(x) \cdot g(x)$\\
     %Sei $A':=\{x \in A| g(x) \neq 0\}$, dann Funktion $\frac{f}{g}: A' \ra \R$ definiert: 
%%%    $\bullet g(x) \neq 0$:$\left(\frac{f}{g}\right)(x) := \frac{f(x)}{g(x)}$\\
%%%    $\bullet f(A) \subseteq B \Ra (g \circ f)(x) := g(f(x))$
%\end{itemize}
\subsection*{Umekehrfunktion}
$f^{-1}: B \ra A$ Umkehrfunktion falls:\\
%\begin{itemize}[leftmargin=*, noitemsep]
    $(f^{-1} \circ f) (x) = f^{-1}(f(x)) = x, x \in A$\\
    $(f \circ f^{-1})(x) = f(f^{-1}(x)) = x, x \in B$
%\end{itemize}
\subsubsection*{Bijektiv-Umkehrfunktion}
$\exists f^{-1}$, $\eq$ $f$ bijektiv.
\subsubsection*{Monotonie Umkehrfunktion}
%$A \subseteq \R$, $f: A \ra B$ Funktion mit $B := f(A) \subseteq \R$. 
$f$ streng monoton $\Ra$ $f^{-1}:B \ra A$ existiert + streng mon. (im g. Sinne)
\subsection*{Beschränktheit}
$f: A \ra B$ heißt nach oben/unten Beschränkt, wenn Bildmenge $f(A)$ oben/unten beschränkt.
\subsection*{Monotonie}
Sei $A \subseteq \R$, $f: A \ra \R$, dann
%\begin{itemize}[leftmargin=*, noitemsep]
    $\bullet$ mon. wachsend: $f(x) \leq f(x')$
    $\bullet$ streng mon. wachs.: $f(x) < f(x')$
    $\bullet$ mon. fallend: $f(x) \geq f(x')$
    $\bullet$ streng mon. fall.: $f(x) > f(x')$
%\end{itemize}
$\forall x, x' \in A$ mit $x < x'$.
\subsection*{Berührpunkt}
$A \subseteq \R$, $a \in \R$, Dann ist $a$ Berührpunkt von $A$, falls $\forall \varepsilon \in \R, \varepsilon > 0 \ \exists b \in  (a - \varepsilon, a + \varepsilon): b \in A$
\subsection*{Grenzwerte Funktionen}
%Sei $f: A\in \R \ra \R$ und $a \in \R$ Berührpunkt von $A$. $\lim_{n \ra \infty} x_n = a \Ra \lim_{n \ra \infty} f(x_n) = c.$\\
%Analog definieren wir:
% $\lim_{x \ra \infty} f(x) = c$, wenn $A$ oben/unten unbeschränkt und $\forall (x_n)$ mit $\lim_{n \ra \infty} x_n = \pm\infty$ gilt $\lim_{n \ra \infty} f(x_n) = c$\\
 %\begin{enumerate}[label=\arabic*., noitemsep, leftmargin=*]
     1. Rechtssei.Grenzw.: 
     $\lim\limits_{n \searrow a} f(x) = x$\\
     %, wenn $a$ Berührpunkt von $A \cap (a, \infty)$ und $\forall$ $(x_n)$ mit $x_n \in A$, $x_n > a$ und $\lim_{n \ra \infty} x_n = a$ gilt: $\lim_{n \ra \infty} f(x_n) = c$
     2. Linkssei. Grenzw.: 
     $\lim\limits_{n \nearrow a} f(x) = x$
     %, wenn $a$ Berührpunkt von $A \cap (a, \infty)$ und $\forall$ $(x_n)$ mit $x_n \in A$, $x_n < a$ und $\lim_{n \ra \infty} x_n = a$ gilt: $\lim_{n \ra \infty} f(x_n) = c$
 %\end{enumerate}
\subsection*{Satz 4.20}
$\lim_{x \ra a} f(x) = f(a) $ $\eq$\\ $\lim_{x \nearrow a} f(x) = \lim_{x \searrow a} f(x) = f(a)$
\subsection*{Stetigkeit}
%Sei $f: A \ra \R$ Funktion,$a \in A$. 
$f$ stetig in $a$, falls $\lim_{x \ra a} f(x) = f(a)$. $f$ stetig, falls $\forall a \in A: f$ stetig
\subsubsection*{\texorpdfstring{$\varepsilon$}{Epsilon}-\texorpdfstring{$\delta$}{Delta}-Kriterium}
%Sei $A \subseteq \R$ und $f: A \ra \R$ funktion. $f$ ist g.d.w. in $a \in A$ stetig, wenn: 
$\forall \varepsilon > 0 \ \exists \delta > 0 \ \forall x \in A: |x- a| < \delta \Ra |f(x) - f(a)| \varepsilon$
%\subsubsection*{Operationen Stetigkeit}
%$f, g: A \ra \R$ in $a \in A$ stetig und $c \in \R$. Dann auch folgendes in $a$ stetig:
%\begin{enumerate}[label=\alph*., noitemsep]
%    $a) f + g: A \ra \R$
%    $b) c \cdot f: A \ra \R$\\
%    $c) f \cdot g: A \ra \R$\\
%    $d) \frac{f}{g}: A' \ra \R$, falls $g(a) \neq 0$\\
%    $e) g \circ f: A \ra \R$, falls $f$ in $a$ und $g$ in $f(a) = b$ stetig
%\end{enumerate}
\subsection*{Zwischenwertsatz}
Sei $f: [a, b] \ra \R$ stetig mit $f(a) \lessgtr 0 \lessgtr f(b)$ $\Ra$ $\exists x \in (a, b)$ mit $f(c)= 0$.\\
Allgemeiner: $\forall y \in\R$: Wenn $f(a) \lessgtr y \lessgtr f(b)$, dann $\exists d \in (a, b):$ $f(d) = y$
\subsection*{Umekehrfunk. stet. Funk.}
%Sei $I \subseteq\R$ Intervall und $f: I \ra \R$ stetig + streng monoton. Dann bildet $f$ $I$ bijektiv auf $f(I)$ ab und $f^{-1}: f(I)\ra \R$ ist stetig.
$f$ stetig + streng mono. $\eq$ $f^{-1}:f(I)\ra \R$ stetig
%\subsection*{Min,Max-kompakt. Interv.}
%Auf $[a, b]$ jede stetige Funktion $f: [a, b] \ra \R$ beschränkt und nimmt Min/Max an.
\subsection*{Gleichmäßige Stetigkeit}
$f: A \ra \R$ gleichmäßig stetig wenn: $\forall \varepsilon > 0 \ \exists \delta > 0 \ \forall x, y \in A: |x - y| < \delta \Ra |f(x) - f(y)| < \varepsilon$\\
$f: A \ra \R$ auf $[a, b] \in A$ stetig $\Ra$ dort auch gleichm. stetig.
\subsection*{Polynom}
Polynomfunktion: $p(x) = a_n x^n + ... + a_1 x + a_0$. $Grad(p) = \max(n)$, wo $a_n \neq 0$
\subsection*{Rationale Funktion}
$p, q$ Polynome, $q(x) \neq 0$, dann ist $r(x) = \left(\frac{p}{q}\right)(x) = \frac{p(x)}{q(x)}$ rat. Funk.
%\subsection*{Polynomdivision}
%\polyset{style=C, div=:,vars=x}
%\polylongdiv{x^2 - x + 1}{x-1}
\subsection*{Linearfaktoren}
$p(x)$ o. Rest d. $q(x) = x - x_1$ teilbar, g.d.w. $x_1 \in \R$ Nullstelle von $p(x)$.
\subsection*{Exponentialfunktion}
$\exp: \R \ra \R_{> 0}$: $\exp(x) = \sum_{k = 0}^{\infty} \frac{x^k}{k!}$
%%%\subsubsection*{Eigenschaften \texorpdfstring{$\exp$}{Exponential}-Funktion}
%\begin{enumerate}[label=\alph*., noitemsep]
%%%    $a) \exp(x + y) = \exp(x) \cdot \exp(y)$\\
%%%    $b) \exp(-x) = \frac{1}{\exp(x)}$\\
%%%    $c) \exp(x) > 0$\\
%%%    $d) \forall n \in \Z: \exp(n) = e^n$\\
%%%    $e) \exp(x) = \lim_{n \ra \infty} (1 + \frac{x}{n})^n$\\
%%%    $f)$ streng mon. wachsend + bijektiv\\
%%%    $g) \lim_{x \ra 0} \frac{\exp(x) - 1}{x} = 1$
%\end{enumerate}
\subsubsection*{1. Satz vom Wachstum}
$\forall n \in \N_{0}$ gilt:
%\begin{itemize}[noitemsep]
    $\bullet \lim_{x \ra \infty} \frac{\exp(x)}{x^n} = \infty$
    $\bullet \lim_{x \ra - \infty} \exp(x) x^n = 0$
%\end{itemize}
\subsubsection*{2. Satz vom Wachstum}
%$\forall n \in \N$ gilt: $\lim_{x \ra \infty} \frac{\ln(x)}{\sqrt[n]{x}} = 0$. 
$\ln(x)$ wächst schwächer als $\sqrt[n]{x}$
%\subsection*{Logarithmus}
%%Umkehrfunktion von $\exp(x)$ ist natürlicher Logarith. $\ln: \R_{>0} \ra \R$
%\subsubsection*{Eigenschaften \texorpdfstring{$\ln(x)$}{des Logarithmus}}
%\begin{enumerate}[label=\alph*., noitemsep]
%    \item $\ln(\exp(x)) = \exp(\ln(x)) = x$
%    \item $\ln(1) = 0$ und $\ln(e) = 1$
%    \item $\ln(x) \begin{cases}
%    < 0 &, x \in (0, 1)\\
%    = 0 &, x = 1\\
%    > 0 &, x > 1
%    \end{cases}$
%    \item $\ln(xy) = \ln(x) + \ln(y)$
%    \item $n \in \Z: \ln(x^n) = n \ln(x)$
%    \item $\ln(x)$ ist stetig
%\end{enumerate}
%%%\subsection*{allgemeine Exp.funktion}
%Sei $a \in \R_{> 0}$.$\exp_a: \R\ra \R: \exp_a(x) := \exp(x \ln(a))$. Schreiben auch $a^x$ statt $\exp_a(x)$.
%%%$\exp_a(x) := \exp(x \ln(a)) = a^x$
%%%\subsection*{Eigenschaft. allg. Potenzen}
%\begin{enumerate}[label=\alph*., noitemsep]
%%%    $a) a^x = \exp_a(x)$ stetig $\forall x \in \R$\\
%%%    $b) \forall n \in \Z$: $\exp_a(x) = a^n$\\
%%%    $c) a^{x+y} = a^x a^y$\\
%%%    $d) (a^x)^y = a^{xy}$\\
%%%    $e) a^x b^x = (ab)^x$\\
%%%    $f) \forall p \in \Z, q \in \N\backslash\{1\}: a^{\frac{p}{q}} = \sqrt[q]{a^p}$
%\end{enumerate}
%%%\subsection*{Log zu allg. Basen}
%%%Sei $a \in \R_{> 0}\backslash\{1\}$, dann $\log_a: \R_{> 0} \ra \R:$ $\log_a(x) := \frac{\ln(x)}{\ln(a)}$
\subsection*{Funktionssymmetrie}
%\begin{itemize}[noitemsep]
    $\bullet$ achsen(gerade): $f(-x) = f(x)$\\
    $\bullet$ punkt(ungerade):$f(-x) = -f(x)$
%\end{itemize}
%%%\subsection*{Hyperbolische Funktionen}
%\begin{itemize}[noitemsep]
%%%    $\bullet \cosh(x) := \frac{e^x + e^{-x}}{2}$\\
%%%    $\bullet \sinh(x) := \frac{e^x - e^{-x}}{2}$\\
%%%    $\bullet \tanh(x) := \frac{\sinh(x)}{\cosh(x)} = \frac{e^x - e^{-x}}{e^{-x}  + e^x}$
%\end{itemize}
%%%\subsubsection*{Eigensch. hyperb. Funkt}
%\begin{enumerate}[label=\alph*., noitemsep]
%%%    $a) \exp(x) = \cosh(x) + \sinh(x)$\\
%%%    $b) \cosh^2(x) - \sinh^2(x) = 1$\\
%%%    $c) \cosh(x) = \sum_{k = 0}^{\infty} \frac{x^{2k}}{(2k)!}$\\
%%%    $d) \sinh(x) = \sum_{k = 0}^{\infty} \frac{x^{2k+1}}{(2k + 1)!}$\\
%%%    $e) \cosh(x + y) = \cosh(x)\cosh(y) + \sinh(x) \sinh(y)$\\
%%%    $f) \sinh(x + y) = \sinh(x) \cosh(y) + \sinh(x) \cosh(y)$
%\end{enumerate}
\subsection*{komplexe \texorpdfstring{$\exp$}{Exponential}-Funktion}
$\C \ra \C$, $\exp(z) = e^z = \sum_{k = 0}^\infty \frac{z^k}{k!}$
%%%\subsection*{Trigonom. Funktionen}
%$\sin/\cos: \R \ra \R$, 
%%%$\tan: \{x| \cos(x) \neq 0\} \ra \R$\\
%\begin{itemize}[noitemsep, leftmargin=*]
    %%%$\bullet \cos(x) := \text{Re}(e^{ix}) = \frac{e^{ix} + e^{-ix}}{2}$\\
    %%%$\bullet \sin(x) := \text{Im}(e^{ix}) = \frac{e^{ix} - e^{-ix}}{2}$\\
    %%%$\bullet \tan(x) := \frac{\sin(x)}{\cos(x)} = \frac{ie^{-ix} - ie^{ix}}{e^{-ix} + e^{ix}}$
%\end{itemize}
%%%\subsubsection*{Eigenschaften trig. Funkt.}
%\begin{enumerate}[label=\alph*., noitemsep]
    %%%$a) \exp(ix) = \cos(x) + i \sin(x)$\\
    %%%$b) \cos^2(x) + \sin^2(x) = 1$\\
    %%%$c) |\sin(x)| \leq 1$ und $|\cos(x)| \leq 1$\\
    %%%$d) \cos(x) = \sum_{k = 0}^\infty (-1)^k \frac{x^{2k}}{(2k)!}$\\
    %%%$e) \sin(x) = \sum_{k = 0}^{\infty} (-1)^k \frac{x^{2k +1}}{(2k + 1)!}$\\
    %%%$f) \cos(x + y) = \cos(x) \cos(y) - \sin(x) \sin(y)$\\
    %%%$g) \sin(x + y) =  \sin(x) \cos(y) + \cos(x)\sin(y)$
%\end{enumerate}
%%%\subsubsection*{Abschätzung Sin-Cos}
%%%Für $x \in (0, 2]$ gilt:\\
%%%%\begin{itemize}[leftmargin=*, noitemsep]
%%%    $\bullet 1 - \frac{x^2}{2} < \cos(x) < 1 - \frac{x^2}{2} + \frac{x^4}{4!}$\\
%%%    $\bullet x - \frac{x^3}{3!} < \sin(x) < x$
%%%%\end{itemize}
%%%\subsubsection*{Folgerung Def. Pi}
%%%\renewcommand{\arraystretch}{1.5}
%%%\begin{tabular}{m{1.1em}m{0.2em}m{0.2em}m{0.2em}m{0.2em}m{0.2em}m{0.2em}m{0.2em}m{0.2em}}
%%%    $x$ & $0$& $\frac{\pi}{6}$ & $\frac{\pi}{4}$ & $\frac{\pi}{3}$ & $\frac{\pi}{2}$ & $\pi$ & $\frac{3\pi}{2}$ & $2\pi$ \\\hline
%%%    $\cos $ & $1$ & $\frac{\sqrt{3}}{2}$       & $\frac{1}{\sqrt{2}}$ & $\frac{1}{2}$        & $0$  & $-1$ & $0$  & $1$ \\\hline
%%%    $\sin $ & $0$ & $\frac{1}{2}$              & $\frac{1}{\sqrt{2}}$ & $\frac{\sqrt{3}}{2}$ & $1$  & $0$  & $-1$ & $0$ \\\hline
%%%    $\tan $ & $0$ & $\frac{1}{\sqrt{3}}$       & $1                 $ & $\sqrt{3}$           & $-$    & $0$   & $-$   & $0$\\
%%%\end{tabular}
%%%%Allgemein gilt für alle $x \in \R$:
%%%$\cos(x + \pi/2) = \sin(x + \pi) =  - \sin(x)$\\
%%%$\cos(x + 2 \pi) = \sin(x + \pi/2) = \cos(x)$
%%%$\cos(x + \pi) = -\cos(x)$\\
%%%$\sin(x + 2\pi) = \sin(x)$
%%%\renewcommand{\arraystretch}{0.5}
%hier ist zu überlegen, ob dies vielleicht weg kann
\subsection*{Periodische Funktionen}
$f: \R \ra \R$ periodische Funktion, wenn $\exists p > 0$, sodass $f(x) = f(x + p), \forall x \in \R$.\\
$\min(p) \in \R_{>0}$ heißt Periode.
%\subsection*{Polarkoordinaten \texorpdfstring{$\C$}{komplexer Zahlen}}
%$\forall z \in \C \ \exists \phi \in \R$, sodass $z = |z|e^{i \phi} = |z| \cos(\phi) + i |z| \sin(\phi)$. Für $z \neq 0$ ist $\phi$ bis auf eine Addition mit Vielfachen von $2 \phi$ eindeutig. Das Paar $(|z|, \phi)$ bezeichnet wir als Polarkoordinaten von $z$ und $\phi$ als Argument von $z$.