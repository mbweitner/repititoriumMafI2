%\usepackage{fancyhdr}
\usepackage[ngerman,german]{babel}
\usepackage[utf8]{inputenc}
%\usepackage[latin1]{inputenc} % für Linux Nutzer
\usepackage{amsmath}
\usepackage{amssymb}
\usepackage{amsthm}
\usepackage{amsfonts}
\usepackage{extarrows}
\usepackage{enumerate}
\usepackage{enumitem}
%%%%%%%%%%%%%%%%%%%%%%%%%%%%%%%%%%%%%%%%%%%%%%%%%%%%%%%%%%%%%%%%%%%%%%%%%%%%%%%%%%%%
%%%%%%%%%%%Weitere Zeichenpakete für zusätzliche Symbole %%%%%%%%%%%%%%%%%%%%%%%%%%%
%%%%%%%%%%%%%%%%%%%%%%%%%%%%%%%%%%%%%%%%%%%%%%%%%%%%%%%%%%%%%%%%%%%%%%%%%%%%%%%%%%%%
\usepackage{stmaryrd}
\usepackage{wasysym}
\usepackage{pifont}
%%%%%%%%%%%%%%%%%%%%%%%%%%%%%%%%%%%%%%%%%%%%%%%%%%%%%%%%%%%%%%%%%%%%%%%%%%%%%%%%%%%%
%Mit dem folgenden Paket können einfache Rechnungen automatisch ausgerechnet werden%
%%%%%%%%%%%%%%%%%%%%%%%%%%%%%%%%%%%%%%%%%%%%%%%%%%%%%%%%%%%%%%%%%%%%%%%%%%%%%%%%%%%%
\usepackage{xfp}%Beispiel: \fpeval{3+5+6} => 14
\usepackage{fp}
\usepackage{spreadtab}%Tabellenberechnungen
\usepackage{tabularx}

\usepackage{graphicx}
\usepackage{tikz}
\usepackage{pgf}
\usepackage{hyperref}
\usepackage{xcolor}
\usepackage{color, colortbl}


%%%%%%%%%%%%%%%%%%%%%%%%%%%%%%%%%%%%%%%%%%%%%%%%%%%%%%%%%%%%%
%%%%%%%%%%%%%% Ganz viele Befehle %%%%%%%%%%%%%%%%%%%%%%%%%%%
%%%%%%%%%%%%%%%%%%%%%%%%%%%%%%%%%%%%%%%%%%%%%%%%%%%%%%%%%%%%%

% Mengen

\newcommand{\N}{\mathbb N}
\newcommand{\R}{\mathbb R}
\newcommand{\C}{\mathbb C}
\newcommand{\Z}{\mathbb Z}
\newcommand{\Q}{\mathbb Q}
\newcommand{\K}{\mathbb K}	

% Ableitungen

\newcommand{\pfrac}[1]{{\frac\partial{\partial #1}}}	% partielle Ableitung als Bruch, Argument ist Variable, nach der abgeleitet wird

% Griechische Buchstaben

\renewcommand{\phi}{\varphi}
\newcommand{\eps}{\varepsilon}

% Pfeile / Äquivalenzen 

\newcommand{\Ra}{\Rightarrow}							% "daraus folgt"
\newcommand{\ra}{\rightarrow}							% "daraus folgt"
\newcommand{\La}{\Leftarrow}							% (analog wie oben)
\newcommand{\la}{\leftarrow}							% (analog wie oben)
\newcommand{\eq}{\Leftrightarrow}						% äquivalent
\newcommand{\upto}{\nearrow}							% Konvergenz von unten
\newcommand{\downto}{\searrow}							% Konvergenz von oben

% Vektorräume

\newcommand{\inv}{{-1}}									% zum schnelleren Invertieren per ^\inv

\newcommand{\mat}[4]{\begin{pmatrix}#1&#2\\#3&#4\end{pmatrix}}
														% 2x2-Matrix
\newcommand{\vek}[2]{\begin{pmatrix}#1\\#2\end{pmatrix}}% 2dim. Vektor

\newcommand{\bpm}{\begin{pmatrix}}						% Matrix (Anfang)
\newcommand{\epm}{\end{pmatrix}}						% Matrix (Ende)

%%%%%%%%%%check und uncheck%%%%%%%%%%%%%%%%
\newcommand{\cmark}{\text{\ding{51}}}                   % Check-Mark
\newcommand{\xmark}{\text{\ding{55}}}                   % UncheckMark
	
\usepackage{polynom}