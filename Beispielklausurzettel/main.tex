\documentclass[a4paper, 8pt]{extarticle} %%%%%%%%%%%%%%%%
%\usepackage{fancyhdr}
\usepackage[ngerman,german]{babel}
\usepackage[utf8]{inputenc}
%\usepackage[latin1]{inputenc} % für Linux Nutzer
\usepackage{amsmath}
\usepackage{amssymb}
\usepackage{amsthm}
\usepackage{amsfonts}
\usepackage{extarrows}
\usepackage{enumerate}
\usepackage{enumitem}
%%%%%%%%%%%%%%%%%%%%%%%%%%%%%%%%%%%%%%%%%%%%%%%%%%%%%%%%%%%%%%%%%%%%%%%%%%%%%%%%%%%%
%%%%%%%%%%%Weitere Zeichenpakete für zusätzliche Symbole %%%%%%%%%%%%%%%%%%%%%%%%%%%
%%%%%%%%%%%%%%%%%%%%%%%%%%%%%%%%%%%%%%%%%%%%%%%%%%%%%%%%%%%%%%%%%%%%%%%%%%%%%%%%%%%%
\usepackage{stmaryrd}
\usepackage{wasysym}
\usepackage{pifont}
%%%%%%%%%%%%%%%%%%%%%%%%%%%%%%%%%%%%%%%%%%%%%%%%%%%%%%%%%%%%%%%%%%%%%%%%%%%%%%%%%%%%
%Mit dem folgenden Paket können einfache Rechnungen automatisch ausgerechnet werden%
%%%%%%%%%%%%%%%%%%%%%%%%%%%%%%%%%%%%%%%%%%%%%%%%%%%%%%%%%%%%%%%%%%%%%%%%%%%%%%%%%%%%
\usepackage{xfp}%Beispiel: \fpeval{3+5+6} => 14
\usepackage{fp}
\usepackage{spreadtab}%Tabellenberechnungen
\usepackage{tabularx}

\usepackage{graphicx}
\usepackage{tikz}
\usepackage{pgf}
\usepackage{hyperref}
\usepackage{xcolor}
\usepackage{color, colortbl}


%%%%%%%%%%%%%%%%%%%%%%%%%%%%%%%%%%%%%%%%%%%%%%%%%%%%%%%%%%%%%
%%%%%%%%%%%%%% Ganz viele Befehle %%%%%%%%%%%%%%%%%%%%%%%%%%%
%%%%%%%%%%%%%%%%%%%%%%%%%%%%%%%%%%%%%%%%%%%%%%%%%%%%%%%%%%%%%

% Mengen

\newcommand{\N}{\mathbb N}
\newcommand{\R}{\mathbb R}
\newcommand{\C}{\mathbb C}
\newcommand{\Z}{\mathbb Z}
\newcommand{\Q}{\mathbb Q}
\newcommand{\K}{\mathbb K}	

% Ableitungen

\newcommand{\pfrac}[1]{{\frac\partial{\partial #1}}}	% partielle Ableitung als Bruch, Argument ist Variable, nach der abgeleitet wird

% Griechische Buchstaben

\renewcommand{\phi}{\varphi}
\newcommand{\eps}{\varepsilon}

% Pfeile / Äquivalenzen 

\newcommand{\Ra}{\Rightarrow}							% "daraus folgt"
\newcommand{\ra}{\rightarrow}							% "daraus folgt"
\newcommand{\La}{\Leftarrow}							% (analog wie oben)
\newcommand{\la}{\leftarrow}							% (analog wie oben)
\newcommand{\eq}{\Leftrightarrow}						% äquivalent
\newcommand{\upto}{\nearrow}							% Konvergenz von unten
\newcommand{\downto}{\searrow}							% Konvergenz von oben

% Vektorräume

\newcommand{\inv}{{-1}}									% zum schnelleren Invertieren per ^\inv

\newcommand{\mat}[4]{\begin{pmatrix}#1&#2\\#3&#4\end{pmatrix}}
														% 2x2-Matrix
\newcommand{\vek}[2]{\begin{pmatrix}#1\\#2\end{pmatrix}}% 2dim. Vektor

\newcommand{\bpm}{\begin{pmatrix}}						% Matrix (Anfang)
\newcommand{\epm}{\end{pmatrix}}						% Matrix (Ende)

%%%%%%%%%%check und uncheck%%%%%%%%%%%%%%%%
\newcommand{\cmark}{\text{\ding{51}}}                   % Check-Mark
\newcommand{\xmark}{\text{\ding{55}}}                   % UncheckMark
	
\usepackage{polynom}
\usepackage[ngerman]{babel}
\usepackage{hyperref}
\hypersetup{
    colorlinks,
    citecolor=black,
    filecolor=black,
    linkcolor=black,
    urlcolor=black
}
\usepackage[
  %showframe,% Seitenlayout anzeigen
  left=0.5cm,
  right=0.5cm,
  top=0.5cm,
  bottom=0.5cm,
]{geometry}

%%%%%%%%%%%%%%%%%%%%%%%%%%%%%%%%%%%%%%%%
%\input{macros} %%%%%%%%%%%%%%%%%%%%%%%%%%%%%%%%%%%%%%%
%%%%%%%%%%%%%%%%%%%%%%%%%%%%%%%%%%%%%%%%%%%%%%%%%%%%%%
%%%%%%%%%%%%%%%%%%%%%%%%%%%%%%%%%%%%%%%%%%%%%%%%%%%%%%
%%%%%%%%%%%%%% NUR hier drunter editieren %%%%%%%%%%%%
%%%%%%%%%%%%%%%%%%%%%%%%%%%%%%%%%%%%%%%%%%%%%%%%%%%%%%

\allowdisplaybreaks
%%%%%%%%%%%%%%%%%%%%%%%%%%%%%%%%%%%%%%%%%%%%%%%%%%%%%%
%% Hier beginnen die Aufgaben %%%%%%%%%%%%%%%%%%%%%%%%
%%%%%%%%%%%%%%%%%%%%%%%%%%%%%%%%%%%%%%%%%%%%%%%%%%%%%%
\usepackage{multicol}
\usepackage{titlesec}
\usepackage{changepage}

\begin{document}
\setlength{\abovedisplayskip}{0.5pt}
\setlength{\belowdisplayskip}{0.5pt}
\renewcommand{\arraystretch}{0.5}
\setlist{leftmargin=0.4cm}
\titlespacing*{\section}
{0pt}{-2pt}{-2pt}
\titlespacing*{\subsection}
{0pt}{-2pt}{-2pt}
\titlespacing*{\subsubsection}
{0pt}{-2pt}{-2pt}



\hfill Mafi Killer 112624
\begin{multicols*}{4}
%\section*{Grundlagen}
%\subsection*{Aussagenlogik}
%
%\subsubsection*{äquivalente Aussagen}
%\begin{adjustwidth}{-0.3cm}{0pt}
%\begin{align*}
%    \begin{array}{l}
%        \neg(\neg A) \eq A \text{ dop. Negation} \\
%        A \land (A \lor B)  \text{ Absorpition} \\
%        \text{Kommutativität} \\
%        A \land B \eq B \land A \\ 
%        A \lor B \eq B \lor A \\ \text{} \\
%        \text{de Morgan} \\
%        \neg (A \land B)    \eq \neg A \lor \neg B \\ 
%        \neg (A \lor B)     \eq \neg A \land \neq B \\ \text{} \\
%        \text{Assoziativität} \\
%        A \lor (B \lor C)   \eq (A \lor B) \lor C \\ 
%        A \lor (B \land C)  \eq (A \lor B) \land (A \lor C) \\ \text{Distributivität} \\
%        A \land (B \land C) \eq (A \land B) \land C \\ \text{} \\
%        A \land (B \lor C)  \eq (A \land B) \lor (A \land C) \\ \text{} \\
%    \end{array}
%\end{align*}
%\end{adjustwidth}
%
%\subsubsection*{All- und Existenzquantor}
%\underline{All-Quantor}: $\forall$\\
%\glqq Für alle $n$ aus $M$ gilt: $A(n)$\grqq\\
%$\forall n \in M: A(n)$\\
%\underline{Existenz-Quantor}:\\
%\glqq Es existiert mindestens ein $n$ aus $M$, für das gilt: $A(n)$\grqq\\
%$\exists n \in M: A(n)$
%\subsection*{Mengen}
%
%\subsubsection*{Mengenverknüpfungen}
%\begin{align*}
%    \begin{array}{l}
%        \text{Vereinigung}\\
%        A\cup B := \{m | m \in A \lor m \in B\}\\
%        \text{Schnitt}\\
%        A \cap B := \{m | m \in A \land m \in B\}\\
%        \text{Differenz}\\
%        A \backslash B := \{m | m \in A \land m \notin B\}\\
%        \text{Kartesisches Produkt}\\
%        A \times B := \{(m, n) | m \in A \land n \in B\}\\
%        \text{Verallgemeinerung Vereinigung}\\
%        \bigcup_{M \in \mathcal{N}} M := \{m | \exists M \in \mathcal{N}: m \in M\}\\
%        \text{Verallgemeinerung Schnitt}\\
%        \bigcap_{M \in \mathcal{N}} M := \{m | \forall M \in \mathcal{N}: m \in M\}\\
%        \text{Komplement}
%        A^c = U \backslash A\\
%    \end{array}
%\end{align*}
%\subsection*{Beweistechniken}
%\subsubsection*{Direkter Beweis}
%Folgerungen Umformungen von bereits bewiesenen Aussagen.\\
%Bei Äquivalenzen müssen beide Richtungen gezeigt werden:\\
%$(A \eq B) \eq ((A \Ra B) \land (B \Ra A))$
%\subsubsection*{Kontraposition}
%Beweis erfolgt indem dei Kontraposition gezeigt wird (rechte Seite)\\
%$(A \Ra B) \eq (\neg B \Ra \neg B)$\\
%\subsubsection*{Widerspruchbeweis}
%Zeige, dass die gegenteilige Aussage zu einer Falschen Aussage führt. Ist dies der Fall, so gilt A.\\
%$((\neg A \Ra C) \land \neq C) \Ra A$
%\subsubsection*{Vollständige Induktion}
%\begin{enumerate}[noitemsep]
%    \item IA: Beweise $A(n = 1)$
%    \item IV: Für ein beliebiges aber festes $n \in \N$ gilt $A(n)$
%    \item IS: Beweise $A(n) \Ra A(n+1)$
%\end{enumerate}




%%%%%%%%%%%%%%%%%%%%%%%%%%%%%%%%%%%%%%%%%%%%%
%%Rechenregeln ohne Ableitung/Integralregeln%
%%%%%%%%%%%%%%%%%%%%%%%%%%%%%%%%%%%%%%%%%%%%%
\section*{Rechenregeln}
\subsection*{Bruchrechnung}
    a) $\frac{a}{b} = \frac{c}{d} \eq ad = bc$
    b) $\frac{ae}{be} = \frac{a}{b}$\\
    c) $\frac{a}{b} \pm \frac{c}{d} = \frac{ad \pm bc}{bd}$
    d) $\frac{\frac{a}{b}}{\frac{e}{d}} = \frac{ad}{be}$
\subsection*{Ungleichungen}
\begin{enumerate}[label=\alph*., noitemsep]
    \item $(a < b) \lor (a > b) \lor (a = b)$
    \item $(a < b) \land (b < c) \Ra a < c$
    \item $(a < b) \land (c \leq d) \Ra a + c < b + d$
    \item $(a < b) \land (x > 0) \Ra ax < bx$\\
          $(a < b) \land (x < 0) \Ra ax > bx$
    \item $a < b \eq a > - b$
    \item $x^2 := x \cdot x > 0$
    \item $0 < a < b \eq 0 < b^{-1} < a^{-1}$
\end{enumerate}
\subsection*{Betrag}
\begin{align*}
    |x| := 
    x \text{, falls } x \geq 0\quad
    -x \text{, falls } x < 0
\end{align*}
\begin{enumerate}[label=\alph*., noitemsep]
    \item $|x| \geq 0 \land (|x| = \eq x = 0)$
    \item $|x \cdot y| = |x| \cdot |y|$
    \item $(|x| < \varepsilon) \eq (x < \varepsilon) \land (-\varepsilon < x) \eq (-\varepsilon < x < \varepsilon)$\\
          $(|x| \leq \varepsilon) \eq (x \leq \varepsilon) \land (-\varepsilon \leq x) \eq (-\varepsilon \leq x \leq \varepsilon)$
    \item $|x + y| \leq |x| + |y| (\text{Dreiecksung.})$
    \item $||x| - |y|| \leq |x - y| (\text{umgekehrte.D.})$
\end{enumerate}
\subsection*{Komplexe Zahlen}
$z_1 + z_2 := (a_1 + a_2, b_1 + b_2)$\\
$z_1 \cdot z_2 := (a_1a_2 - b_1b_2, a_1b_2 + a_2b_1)$\\
Nullelement $(0,0)$
Einselement $(1, 0)$\\
$-(a, b) = (-a, -b)$(Negativelement)\\
$(a, b)^{-1} = \left(\frac{a}{a^2 + b^2}, \frac{-b}{a^2 + b^2}\right)$ (Invers.)\\
$\bar z := (a, -b) = a - ib$\\
$|z| := \sqrt{z \bar z} = \sqrt{a^2 + b^2}$\\
$d(z_1, z_2) = |z_1 - z_2|$\\
$\overline{z_1 \cdot z_2} = \bar z_1 \dot \bar z_2 \quad\quad$
$\overline{z_1 + z_2} = \bar z_1 + \bar z_2$
\subsection*{Potenzregeln}
$a^n a^m = a^{n + m}\quad$
$(a^n)^m = a^{nm}\quad$
$a^n b^n = (a \cdot b)^n$
\subsection*{Algemeine Potenzen}
$\exp_a(x) := \exp(x \ln(a)) = a^x$\\
$a) a^x = \exp_a(x)$ stetig $\forall x \in \R$\\
$b) \forall n \in \Z$: $\exp_a(x) = a^n$\\
$c) a^{x+y} = a^x a^y$\\
$d) (a^x)^y = a^{xy}$\\
$e) a^x b^x = (ab)^x$\\
$f) \forall p \in \Z, q \in \N\backslash\{1\}: a^{\frac{p}{q}} = \sqrt[q]{a^p}$
\subsection*{Binomialkoeffizient}
\begin{align*}
    &\binom{n}{k} := \begin{cases}
    \frac{n!}{(n - k)! \cdot k!} = \\
    \frac{n \cdot (n-1)\cdot ... \cdot (n-k +1)}{k \cdot (k-1) \cdot ... \cdot 1} & n\geq k\\
    0 & n < k
    \end{cases}
\end{align*}
$\forall n, k \in \N_0: \binom{n}{k} + \binom{n}{k+1} = \binom{n +1}{k+1}$
\subsection*{Folgen}
Seien $(a_n)_{n \in \N}, (b_n)_{n \in \N}$ konvergente Folgen und $c\in \R$
\begin{enumerate}[label=\alph*., noitemsep]
    \item $(a_n) + (b_n) = (a_n + b_n)$
    \item $c \cdot (a_n) = (c \cdot a_n)$
    \item $(a_n) \cdot (b_n) = (a_n \cdot b_n)$
    \item $\frac{(a_n)}{(b_n)} = \left( \frac{a_n}{b_n} \right)$, falls $b_n \neq 0$
    \item $\lim_{n \ra \infty} (a_n + b_n) = \lim_{n \ra \infty} (a_n) + \lim_{n \ra \infty} (b_n)$
    \item $\lim_{n \ra \infty} c \cdot (a_n) = c \cdot \lim_{n \ra \infty} (a_n)$
    \item $\lim_{n \ra \infty} (a_n) \cdot (b_n) = \lim_{n \ra \infty} (a_n) \cdot \lim_{n \ra \infty} (b_n)$
    \item $\lim_{n \ra \infty} \frac{(a_n)}{(b_n)} = \frac{\lim_{n \ra \infty} (a_n)}{\lim_{n \ra \infty} (b_n)}$, falls $b_n \neq 0$ und $\lim_{n \ra \infty} b_n \neq 0$
\end{enumerate}
\subsection*{Konv. Reihen}
\begin{enumerate}[label=\alph*., noitemsep]
    \item $\sum_{k = 1}^\infty (a_k \pm b_k)$ konvergent. Für die Grenzwerte gilt:$\sum_{k = 1}^\infty (a_k \pm b_k) = \sum_{k = 1}^\infty a_k \pm \sum_{k = 1}^\infty b_k$
    \item $\sum_{k = 1}^\infty c \cdot a_k$ konvergent für $c \in \R$. Es gilt:$\sum_{k = 1}^\infty c \cdot a_k = \cdot \sum_{k = 1}^\infty a_k$
    \item $\forall l \in N \ l > 0: \sum_{k = l}^\infty a_k$ konvergiert $\eq \sum_{k = 1}^\infty a_k$ konvergiert
    \item Gilt $a_k \leq b_k \forall k \in \N: \sum_{k = 1}^\infty a_k \leq \sum_{k = 1}^\infty b_k$
\end{enumerate}
\subsection*{Funktionen}
$\bullet (f + g)(x) := f(x) + g(x)$\\
$\bullet (cf)(x) := c f(x)$\\
$\bullet (f \cdot g)(x) := f(x) \cdot g(x)$\\
$\bullet g(x) \neq 0$:$\left(\frac{f}{g}\right)(x) := \frac{f(x)}{g(x)}$\\
$\bullet f(A) \subseteq B \Ra (g \circ f)(x) := g(f(x))$
\subsection*{Stetigkeit erhalten}
$a) f + g: A \ra \R$
$b) c \cdot f: A \ra \R$\\
$c) f \cdot g: A \ra \R$\\
$d) \frac{f}{g}: A' \ra \R$, falls $g(a) \neq 0$\\
$e) g \circ f: A \ra \R$, falls $f$ in $a$ und $g$ in $f(a) = b$ stetig
\subsection*{Exponentialfunktion}
$a) \exp(x + y) = \exp(x) \cdot \exp(y)$\\
$b) \exp(-x) = \frac{1}{\exp(x)}$\\
$c) \exp(x) > 0$\\
$d) \forall n \in \Z: \exp(n) = e^n$\\
$e) \exp(x) = \lim_{n \ra \infty} (1 + \frac{x}{n})^n$\\
$f)$ streng mon. wachsend + bijektiv\\
$g) \lim_{x \ra 0} \frac{\exp(x) - 1}{x} = 1$
\subsection*{Logarithmus}
\begin{enumerate}[label=\alph*., noitemsep]
    \item $\ln(\exp(x)) = \exp(\ln(x)) = x$
    \item $\ln(1) = 0$ und $\ln(e) = 1$
    \item $\ln(x) \begin{cases}
    < 0 &, x \in (0, 1)\\
    = 0 &, x = 1\\
    > 0 &, x > 1
    \end{cases}$
    \item $\ln(xy) = \ln(x) + \ln(y)$
    \item $n \in \Z: \ln(x^n) = n \ln(x)$
    \item $\ln(x)$ ist stetig
\end{enumerate}
\subsubsection*{Logarithmus zu alg. Basis}
Sei $a \in \R_{> 0}\backslash\{1\}$, dann $\log_a: \R_{> 0} \ra \R:$ $\log_a(x) := \frac{\ln(x)}{\ln(a)}$
\subsection*{Sinh-Cosh}
$\bullet \cosh(x) := \frac{e^x + e^{-x}}{2}$\\
$\bullet \sinh(x) := \frac{e^x - e^{-x}}{2}$\\
$\bullet \tanh(x) := \frac{\sinh(x)}{\cosh(x)} = \frac{e^x - e^{-x}}{e^{-x}  + e^x}$\\
$a) \exp(x) = \cosh(x) + \sinh(x)$\\
$b) \cosh^2(x) - \sinh^2(x) = 1$\\
$c) \cosh(x) = \sum_{k = 0}^{\infty} \frac{x^{2k}}{(2k)!}$\\
$d) \sinh(x) = \sum_{k = 0}^{\infty} \frac{x^{2k+1}}{(2k + 1)!}$\\
$e) \cosh(x + y) = \cosh(x)\cosh(y) + \sinh(x) \sinh(y)$\\
$f) \sinh(x + y) = \sinh(x) \cosh(y) + \sinh(x) \cosh(y)$
\subsection*{Sin-Cos}
$\tan: \{x| \cos(x) \neq 0\} \ra \R$\\
$\bullet \cos(x) := \text{Re}(e^{ix}) = \frac{e^{ix} + e^{-ix}}{2}$\\
$\bullet \sin(x) := \text{Im}(e^{ix}) = \frac{e^{ix} - e^{-ix}}{2}$\\
$\bullet \tan(x) := \frac{\sin(x)}{\cos(x)} = \frac{ie^{-ix} - ie^{ix}}{e^{-ix} + e^{ix}}$\\
$a) \exp(ix) = \cos(x) + i \sin(x)$\\
$b) \cos^2(x) + \sin^2(x) = 1$\\
$c) |\sin(x)| \leq 1$ und $|\cos(x)| \leq 1$\\
$d) \cos(x) = \sum_{k = 0}^\infty (-1)^k \frac{x^{2k}}{(2k)!}$\\
$e) \sin(x) = \sum_{k = 0}^{\infty} (-1)^k \frac{x^{2k +1}}{(2k + 1)!}$\\
$f) \cos(x + y) = \cos(x) \cos(y) - \sin(x) \sin(y)$\\
$g) \sin(x + y) =  \sin(x) \cos(y) + \cos(x)\sin(y)$
\subsubsection*{Abschätzung Sin-Cos}
Für $x \in (0, 2]$ gilt:\\
%\begin{itemize}[leftmargin=*, noitemsep]
    $\bullet 1 - \frac{x^2}{2} < \cos(x) < 1 - \frac{x^2}{2} + \frac{x^4}{4!}$\\
    $\bullet x - \frac{x^3}{3!} < \sin(x) < x$
%\end{itemize}
\subsubsection*{Folgerung Def. Pi}
\renewcommand{\arraystretch}{1.5}
\begin{tabular}{m{1.1em}m{0.2em}m{0.2em}m{0.2em}m{0.2em}m{0.2em}m{0.2em}m{0.2em}m{0.2em}}
    $x$ & $0$& $\frac{\pi}{6}$ & $\frac{\pi}{4}$ & $\frac{\pi}{3}$ & $\frac{\pi}{2}$ & $\pi$ & $\frac{3\pi}{2}$ & $2\pi$ \\\hline
    $\cos $ & $1$ & $\frac{\sqrt{3}}{2}$       & $\frac{1}{\sqrt{2}}$ & $\frac{1}{2}$        & $0$  & $-1$ & $0$  & $1$ \\\hline
    $\sin $ & $0$ & $\frac{1}{2}$              & $\frac{1}{\sqrt{2}}$ & $\frac{\sqrt{3}}{2}$ & $1$  & $0$  & $-1$ & $0$ \\\hline
    $\tan $ & $0$ & $\frac{1}{\sqrt{3}}$       & $1                 $ & $\sqrt{3}$           & $-$    & $0$   & $-$   & $0$\\
\end{tabular}
%Allgemein gilt für alle $x \in \R$:
$\cos(x + \pi/2) = \sin(x + \pi) =  - \sin(x)$\\
$\cos(x + 2 \pi) = \sin(x + \pi/2) = \cos(x)$
$\cos(x + \pi) = -\cos(x)$\\
$\sin(x + 2\pi) = \sin(x)$
\renewcommand{\arraystretch}{0.5}%%%



\section*{Zahlenmengen}
%\subsection*{Körper}
%%%%%%%%%%%%%%%%%%%%%%%%%%%%%%%%%Hier nochmal drüber schauen viellecht passt es ja %%%%%%%%%%%%%%%%%%%%%%%%%%%%%
%Hier könnten vielleicht noch die Körperaxiome und die Folgerungen daraus hin. 
%%%%%%%%%%%%%%%%%%%%%%%%%%%%%%%%%Hier nochmal drüber schauen viellecht passt es ja %%%%%%%%%%%%%%%%%%%%%%%%%%%%%
%%%\subsection*{Bruchrechenregeln}
%\begin{enumerate}[label=\alph*., noitemsep]
%%%    a) $\frac{a}{b} = \frac{c}{d} \eq ad = bc$
%%%    b) $\frac{ae}{be} = \frac{a}{b}$\\
%%%    c) $\frac{a}{b} \pm \frac{c}{d} = \frac{ad \pm bc}{bd}$
%%%    d) $\frac{\frac{a}{b}}{\frac{e}{d}} = \frac{ad}{be}$
%\end{enumerate}
%%%\subsubsection*{Folgerungen für Ungleichungen}
%In geordneten Körper $K$ gilt für $a, b, c, d \in K$ und $x \in K\backslash \{0\}$
%%%\begin{enumerate}[label=\alph*., noitemsep]
%%%    \item $(a < b) \lor (a > b) \lor (a = b)$
%%%    \item $(a < b) \land (b < c) \Ra a < c$
%%%    \item $(a < b) \land (c \leq d) \Ra a + c < b + d$
%%%    \item $(a < b) \land (x > 0) \Ra ax < bx$\\
%%%          $(a < b) \land (x < 0) \Ra ax > bx$
%%%    \item $a < b \eq a > - b$
%%%    \item $x^2 := x \cdot x > 0$
%%%    \item $0 < a < b \eq 0 < b^{-1} < a^{-1}$
%%%\end{enumerate}
%%%\subsection*{Betrag und Folgerungen}
%%%\begin{align*}
%%%    |x| := 
%%%    x \text{, falls } x \geq 0\quad
%%%    -x \text{, falls } x < 0
%%%\end{align*}
%%%\begin{enumerate}[label=\alph*., noitemsep]
%%%    \item $|x| \geq 0 \land (|x| = \eq x = 0)$
%%%    \item $|x \cdot y| = |x| \cdot |y|$
%%%    \item $(|x| < \varepsilon) \eq (x < \varepsilon) \land (-\varepsilon < x) \eq (-\varepsilon < x < \varepsilon)$\\
%%%          $(|x| \leq \varepsilon) \eq (x \leq \varepsilon) \land (-\varepsilon \leq x) \eq (-\varepsilon \leq x \leq \varepsilon)$
%%%    \item $|x + y| \leq |x| + |y| (\text{Dreiecksung.})$
%%%    \item $||x| - |y|| \leq |x - y| (\text{umgekehrte.D.})$
%%%\end{enumerate}
\subsection*{Metrik}
%Sei $A$ eine Menge. Wir nennen eine Abbildung $d:A \times A \ra \R$ eine Metrik auf $A$, wenn für alle $x, y, z \in A$ drei Eigenschaften erfüllt sind:
%\begin{enumerate}[label=\alph*., noitemsep]
    Positive Definitheit:\\
        $d(x, y) > 0 \text{ für } x \neq y$\\
        $d(x, y) = 0 \text{ für } x = y$\\
    Symmetrie: $d(x, y) = d(y, x)$
    Dreiecksunglei.: $d(x, y) \leq d(x, z) + d(z, y)$
%\end{enumerate}
%Beispiel: $d:\R \times \R \ra \R$ mit $d(x, y) = |x - y|$
%\subsection*{Gaußklammern}
%Sei $x \in \R$ und $m, n \in \Z$:
%\begin{enumerate}[label=\alph*), noitemsep]
%    \item $m \leq x < m+1$
%    \item $n - 1 < x \leq n$
%\end{enumerate}
%$\lfloor x \rfloor := m\quad\quad$ $\lceil x \rceil := n$
%\subsection*{Modulo}
%Sei $m, r, z \in \Z$ und $n \in \N$:
%$z = mn + r$\\
%$r = z \mod n$
%%%\subsection*{Komplexe Zahlen}

%\subsubsection*{Definition}
%$\C := \{(a, b) \in \R \times \R\}$
%Für $z = (a, b)$ ist $Re(z) := a$ der Realteil und $Im(z) := b$ der Imaginärteil
%%%\subsubsection*{Rechnen mit \texorpdfstring{$\C$}{Komplexen Zahlen}}
%%%$z_1 + z_2 := (a_1 + a_2, b_1 + b_2)$\\
%%%$z_1 \cdot z_2 := (a_1a_2 - b_1b_2, a_1b_2 + a_2b_1)$\\
%%%Nullelement $(0,0)$
%%%Einselement $(1, 0)$\\
%%%$-(a, b) = (-a, -b)$(Negativelement)\\
%%%$(a, b)^{-1} = \left(\frac{a}{a^2 + b^2}, \frac{-b}{a^2 + b^2}\right)$ %%%(Invers.)
%%%\subsubsection*{Konjugation und Betrag}
%%%$\bar z := (a, -b) = a - ib$\\
%%%$|z| := \sqrt{z \bar z} = \sqrt{a^2 + b^2}$
%%%\subsubsection*{Konjug.eigenschaften + Metrik}
%%%$d(z_1, z_2) = |z_1 - z_2|$\\
%%%$\overline{z_1 \cdot z_2} = \bar z_1 \dot \bar z_2 \quad\quad$
%%%$\overline{z_1 + z_2} = \bar z_1 + \bar z_2$
%\subsection*{Summen, Produkte, ...}
%\subsubsection*{Def. Summe, Produkt}
%\begin{align*}
%    &\sum_{k = m}^n a_k := \begin{cases}
%    a_m + a_{m+1} +  \\
%    ... + a_n & m \leq n\\
%    0 &\text{,sonst}
%    \end{cases}\\
%    &\prod_{k = m}^{n} a_k :=\begin{cases}
%    a_m \cdot a_{m+1} \cdot  \\
%    ... \cdot a_n & m \leq n\\
%    1 &\text{, sonst}
%    \end{cases}
%\end{align*}
%\subsubsection*{Potenzen}
%$x^n := \prod\limits_{k = 1}^n x$\\
%Für $x \neq 0$: 
%$x^{-n} := \frac{1}{x^n} \quad\quad x^0 := 1$
%\subsubsection*{Rechenregeln Potenzen}
%$a^n a^m = a^{n + m}\quad$
%$(a^n)^m = a^{nm}\quad$
%$a^n b^n = (a \cdot b)^n$
\subsubsection*{Fakultät}
$n! := \prod\limits_{k=1}^n k = 1 \cdot 2 \cdot ... \cdot n \quad\quad 0! = 1$\\
$(n!)_{n\in\N} = (1), 1, 2, 6, 24, 120, 720$
%%%\subsubsection*{Binomialkoeffizient}
%%%\begin{align*}
%%%    &\binom{n}{k} := \begin{cases}
%%%    \frac{n!}{(n - k)! \cdot k!} = \\
%%%    \frac{n \cdot (n-1)\cdot ... \cdot (n-k +1)}{k \cdot (k-1) \cdot ... \cdot 1} & n\geq k\\
%%%    0 & n < k
%%%    \end{cases}
%%%\end{align*}
%%%$\forall n, k \in \N_0: \binom{n}{k} + \binom{n}{k+1} = \binom{n +1}{k+1}$
\subsection*{Binomischer Lehrsatz}
Für $a, b \in \R$ und $n \in \N$:
$(a + b)^n = \sum\limits_{k = 0}^n \binom{n}{k} a^{n - k}b^k$
\subsection*{Bernoullische Ungleichung}
$\forall x \in \R, x \geq -1, n \in \N_0$ gilt:
$(1 + x)^n \geq 1 + nx$
\subsubsection*{Satz 2.49}
$\forall x \in \R$ mit $x \geq 0$ und $\forall n \in N$ mit $n \geq 2$ gilt
$(1 + x)^n \geq \frac{n^2 x^2}{4}$
\section*{Folgen}
%\subsection*{Definition}
%Eine Folge ist eine Abbildung, bei der jedem $n \in \N$ ein $a_n\in\R$ zugeordnet %wird. Schreibweise: $(a_n)$ oder $(a_n)_{n \in \N}$
\subsection*{Konvergenz, Divergenz}
Folge $(a_n)$ ist konvergent, wenn gilt:
$\forall \varepsilon > 0 \exists n_0 \in \N \ \forall n \geq n_0: |a_n -a | < \varepsilon$\\
%\begin{enumerate}[label=\arabic*., noitemsep]
%    \item nicht konvergent $\Ra$ divergent
%    \item Falls $(a_n)$ gegen $a$ konvergiert, so ist $a$ Grenzwert von $(a_n)$. Schreibweise: $\lim_{n \rightarrow \infty} a_n = a$ oder $a_n \ra a$ für $n \ra \infty$
%    \item Falls $\lim_{n \rightarrow \infty} a_n = 0 \Ra$ Nullfolge 
%\end{enumerate}
\subsubsection*{Folgengrenzwert ist eindeutig}
%Der Grenzwert einer Folge ist, falls er existiert eindeutig!
\subsubsection*{Divergenz inverser Nullfolge}
Ist Folge $(a_n)$ Nullfolge mit $a_n \neq 0$, dann ist Folge $(b_n) = \frac{1}{a_n}$ divergent.
\subsubsection*{Bestimmte Divergenz}
Folge $(a_n)$ ist bestimmt divergent gegen $\infty/-\infty$, wenn $b_n = \frac{1}{a_n}$ eine Nullfolge ist und $\exists n_0 \in \N \forall n \geq n_0: a_n \lessgtr 0$. %Wir schreiben:\\
%$\lim_{n \ra \infty} a_n = \infty/-\infty$
%\subsubsection*{Beschränkte Folge}
%Folge $(a_n)$ nach oben (unten) beschränkt, wenn Menge $M = \{a_n | n \in \N\}$ nach oben (unten) beschränkt ist.\\
%Ist $(a_n)$ nach oben und unten beschränkt so heißt sie beschränkt.
\subsubsection*{Konvergenz \texorpdfstring{$\Rightarrow$}{folglich} Beschränkt}
%Jede Konvergente Folge ist beschränkt.
\subsection*{Monotonie}
%Folge $(a_n)$ heißt:
\begin{itemize}[noitemsep, label={}, leftmargin=*]
    \item monoton wachsend: 
    $a_n \leq a_{n+1}$
    \item streng m. wachsend:
    $a_n < a_{n+1}$
    \item monoton fallend:
    $a_n \geq a_{n+1}$
    \item streng m. fallend:
    $a_n > a_{n+1}$
\end{itemize}
\subsubsection*{\small{Mono. \texorpdfstring{$+$}{plus} Beschrä. \texorpdfstring{$\Rightarrow$}{folglich} Konvergenz}}
%Jede beschränkte montone Folge ist konvergent.
\begin{enumerate}[label=\alph*., noitemsep]
    \item $(a_n)$ monoton wachsend $+$ oben beschränkt $\Ra$ konvergent. Es gilt $\lim_{n\ra \infty} a_n= \sup \{a_n|n \in \N\}$
    \item $(a_n)$ monoton fallend $+$ unten beschränkt $\Ra$ konvergent. Es gilt $\lim_{n\ra \infty} a_n= \inf \{a_n|n \in \N\}$
\end{enumerate}
%%%\subsection*{Rechenregeln}
%%%Seien $(a_n)_{n \in \N}, (b_n)_{n \in \N}$ konvergente Folgen und $c\in \R$
%%%\begin{enumerate}[label=\alph*., noitemsep]
%%%    \item $(a_n) + (b_n) = (a_n + b_n)$
%%%    \item $c \cdot (a_n) = (c \cdot a_n)$
%%%    \item $(a_n) \cdot (b_n) = (a_n \cdot b_n)$
%%%    \item $\frac{(a_n)}{(b_n)} = \left( \frac{a_n}{b_n} \right)$, falls $b_n \neq 0$
%%%    \item $\lim_{n \ra \infty} (a_n + b_n) = \lim_{n \ra \infty} (a_n) + \lim_{n \ra \infty} (b_n)$
%%%    \item $\lim_{n \ra \infty} c \cdot (a_n) = c \cdot \lim_{n \ra \infty} (a_n)$
%%%    \item $\lim_{n \ra \infty} (a_n) \cdot (b_n) = \lim_{n \ra \infty} (a_n) \cdot \lim_{n \ra \infty} (b_n)$
%%%    \item $\lim_{n \ra \infty} \frac{(a_n)}{(b_n)} = \frac{\lim_{n \ra \infty} (a_n)}{\lim_{n \ra \infty} (b_n)}$, falls $b_n \neq 0$ und $\lim_{n \ra \infty} b_n \neq 0$
%%%\end{enumerate}
%\subsection*{Größenvergl. konv. Folgen}
%Seien $(a_n), (b_n)$ konvergente Folgen mit $(a_n) \leq (b_n)$ Dann gilt: $\lim_{n \ra \infty} (a_n) \leq \lim_{n \ra \infty} (b_n)$
\subsection*{Sandwich-Theorem}
%Seien $(a_n), (b_n), (c_n)$ Folgen, für die $\exists n_0$, sodass $n \geq n_0$ gilt: $(a_n) \leq (b_n) \leq (c_n)$.\\
Wenn $(a_n), (c_n)$ konvergent und gilt $\lim_{n \ra \infty} (a_n) = \lim_{n \ra \infty} (c_n)$, dann ist auch $(b_n)$ konvergent und es gilt:\\
$\lim\limits_{n \ra \infty} (a_n) = \lim\limits_{n \ra \infty} (b_n) = \lim\limits_{n \ra \infty} (c_n)$
\subsection*{Teilfolgen}
\subsubsection*{Grenzwert Teilfolge}
Jede Teilfolge $(a_{n_k})$ einer konvergenten Folge $(a_n)$ ist konvergent. Es gilt: $\lim_{k \ra \infty} a_{n_k} = \lim_{n \ra \infty} a_n = a$
\subsubsection*{Divergenz durch Teilfolge}
Folge $(a_n)$ divergent, wenn:
\begin{enumerate}[label=\alph*., noitemsep]
    \item 1 divergente Teilfolge
    \item 2 konverg. Teilf. $(a_{n_k}), (a_{n_l})$ mit $\lim_{k \ra \infty} \neq \lim_{l \ra \infty} (a_{n_l}$
\end{enumerate}
\subsubsection*{Jede Fol. hat mon. Teilf.}
%\subsubsection*{Satz 3.29}
%Jede Folge enhtält eine monotone Teilfolge
\subsubsection*{Balzano-Weierstraß}
Jede beschränkte Folge $(a_n)$ besitzt eine konvergente Teilfolge.
\subsubsection*{Häufungspunkt}
Für $(a_n)$ heißt $a$ Häufungspunkt, wenn Teilfolge $(a_{n_k})$ existiert und $\lim_{k \ra \infty} (a_{n_k}) = a$
\subsection*{Cauchy-Folge}
%$(a_n)$ heißt Cauchy-Folge, wenn gilt:\\
$\forall \varepsilon > 0 \exists n_0 \in \N \forall n > n_0:$ $|a_n - a_{n_0}| < \varepsilon$
\subsubsection*{Satz 3.34}
%Folge ist genau dann konvergent, wenn sie eine Cauchy-Folge ist. Das bedeutet:
\begin{enumerate}[label=\alph*., noitemsep]
    \item $\forall$ konverg. Folge $\Ra$ Cauchy-Folge
    \item Jede Cauchy-Folge ist konvergent
\end{enumerate}
\subsection*{Intervallle}
\subsubsection*{Kompaktheit}
$I$ kompakt$\eq$abgeschl. + beschränkt
\subsubsection*{Intervalschachtelung}
Folge $(I_n)$ von abgeschl. Intervallen $I_n$ heißt Intervallschachtelung, wenn:
\begin{itemize}[noitemsep]
    \item $\forall n \in \N: I_{n+1} \subset I$
    \item $\lim_{n \ra \infty} |I_n| = 0$
\end{itemize}
\subsubsection*{Konvergenz Intervalschacht.}
Für jede Intervallschach. $(I_n)$ existiert ein eindeutiges $x \in \R$, für das: $x \in I, \forall n \in \N$\\
$(I_n)$ konvergiert gegen $x$.
\section*{Reihen}
\subsection*{Def. Reihe konvergent}
%$\sum_{k = 1}^\infty a_k = a_1 + ...$ eine Reihe. $s_n = \sum_{k = 1}^{n} a_k$ die n-te Teilsumme der Reihe. 
Folge der Teilsummen konvergent $\Ra$ Reihe konvergent. Sonst divergent.
\subsection*{Cauchy-Konvergenzkrit.}
Reihe konvergiert g.d.w. gilt: $\forall \varepsilon > 0 \exists n_0\in \N \forall n \geq m \geq n_0: \left|\sum\limits_{k = m}^{n} < \varepsilon \right|$ 
\subsection*{Notw. Konvergenzkrit.}
$\sum_{k=1}^\infty a_k$ konvergente Reihe $\Ra$ Folge $(a_k)$ ist Nullfolge $\Ra \lim_{k\ra \infty} a_k = 0$.
\subsection*{Teilsummenbeschränktheit}
$\sum_{k=1}^\infty a_k$ mit $a_k \geq 0$ $\forall k \in \N$ konvergiert g.d.w. Folge der Teilsummen beschränkt.
%%%\subsection*{Rechenregeln konv. Reihen}
%$\sum_{k=1}^\infty a_k$, $\sum_{k = 1}^\infty b_k$ konverg. Reihen:
%%%\begin{enumerate}[label=\alph*., noitemsep]
%%%    \item $\sum_{k = 1}^\infty (a_k \pm b_k)$ konvergent. Für die Grenzwerte gilt:$\sum_{k = 1}^\infty (a_k \pm b_k) = \sum_{k = 1}^\infty a_k \pm \sum_{k = 1}^\infty b_k$
%%%    \item $\sum_{k = 1}^\infty c \cdot a_k$ konvergent für $c \in \R$. Es gilt:$\sum_{k = 1}^\infty c \cdot a_k = \cdot \sum_{k = 1}^\infty a_k$
%%%    \item $\forall l \in N \ l > 0: \sum_{k = l}^\infty a_k$ konvergiert $\eq \sum_{k = 1}^\infty a_k$ konvergiert
%%%    \item Gilt $a_k \leq b_k \forall k \in \N: \sum_{k = 1}^\infty a_k \leq \sum_{k = 1}^\infty b_k$
%%%\end{enumerate}
\subsection*{Def. absolute Konvergenz}
$\sum_{k = 1}^\infty a_k$ abs. k. $\eq \sum_{k = 1}^\infty |a_k|$ konv.
\subsubsection*{Reihenumordnung}
$\sum_{k = 1}^\infty a_k$ abs. konv. $\Ra$ Jede Umordnung konverg. gegen selben Grenzw.
\subsubsection*{abs. Konv. \texorpdfstring{$\Ra$}{folgt} Konvergenz}
$\sum_{k = 1}^\infty a_k$ abs. konv. $\Ra$ konvergent
\subsection*{Cauchy-Produkt}
$\sum_{k = 0}^\infty a_k, \sum_{k = 0}^\infty b_k$ abs. konverg..Für $n \in \N$ sei $c_n := \sum_{k = 0}^n a_k \cdot b_{n - k}$, dann ist $\sum_{k = 0}^\infty = \left(\sum_{k = 0}^\infty a_k\right) \cdot \left(\sum_{k = 0}^\infty b_k\right)$ abs. konv.
\subsection*{Konvergenzkriterien}

\subsubsection*{Leibnitz-Kriterium}
$(a_k)$ monoton fallende Folge mit $\forall k \in \N a_k \geq 0$ mit $\lim_{k \ra \infty}a_k = 0$, dann konvergiert $\sum_{k = 1}^\infty (-1)^k a_k$.
\subsubsection*{Majorantenkriterium}
$\sum_{k = 1}^\infty c_k$ konvergent mit $\forall k \in \N: c_k \geq 0$. Wenn für $\sum_{k = 1}^\infty a_k \exists k_0 \in \N$, sodass $\forall k \geq k_0$ gilt $|a_k| \leq c_k$, dann konvergiert $\sum_{k = 1}^\infty a_k$ absolut.
\subsubsection*{Minorantenkriterium}
$\sum_{k = 1}^\infty c_k$ divergent mit $\forall k \in \N: c_k \geq 0$. Wenn für $\sum_{k = 1}^\infty a_k \exists k_0 \in \N$, sodass $\forall k \geq k_0$  gilt $a_k \geq c_k$, dann divergiert $\sum_{k = 1}^\infty a_k$
\subsubsection*{Wurzelkriterium}
%hier könnte nur die Limesform ausreichen %entfernen
%\begin{enumerate}
%    \item Wenn festes $q \in \R$ mit $0 < q < 1$ und $k_0 \in \N$ existiert, sodass $\forall k \geq k_0: \sqrt[k]{|a_k|} \leq q$, dann konvergiert $\sum_{k = 1}^\infty a_k$ absolut
%    \item $\exists k_0 \in \N$, sodass $\forall k \geq k_0: \sqrt[k]{|a_k|} \geq 1$, dann divergiert $\sum_{k = 1}^\infty a_k$
%\end{enumerate}
%Limesform:\\
$\exists a = \lim_{k \ra \infty} \sqrt[k]{|a_k|}$, dann gilt:
%\begin{itemize}[leftmargin = *, noitemsep]
     $\bullet a < 1 \Ra$ absolut konvergent\\
     $\bullet a > 1 \Ra$ divergent\\
     $\bullet a = 1 \Ra$ unwissend
%\end{itemize}
\subsubsection*{Quotientenkriterium}
%\begin{enumerate}[label=\alph*., noitemsep]
%    \item Wenn festes $q \in \R$ mit $0 < q < 1$ und $k_0 \in \N$ existiert, sodass $\forall k \geq k_0: a_k \neq 0 \land \left| \frac{a_{k+1}}{a_k} \right| \leq q$, dann konvergiert $\sum_{k = 1}^\infty a_k$ absolut.
%    \item $\exists k_0 \in \N$, sodass $\forall k \geq k_0: a_k \neq 0 \land \left| \frac{a_{k+1}}{a_k} \right| \geq 1$, dann divergiert $\sum_{k = 1}^\infty a_k$
%\end{enumerate}
%Limesform:\\
$\exists a = \lim_{k \ra \infty} \left| \frac{a_{k+1}}{a_k} \right|$, dann gilt:
%\begin{itemize}[leftmargin=*, noitemsep]
    $\bullet a < 1 \Ra$ konvergiert absolut\\
    $\bullet a > 1 \Ra$ divergiert\\
    $\bullet a = 1 \Ra $ unwissend 
%\end{itemize}
\subsection*{Potenzreihen}
\subsubsection*{Definition}
%Folge $(a_k)$, $x, x_0 \in \R$, dann Potenzreihe $P(x, x_0)$ mit Entwicklungspunkt $x_0$ definiert als: 
$P(x, x_0) = \sum_{k = 0}^\infty a_k \cdot (x - x_0)^k$\\
$x_0 = 0$, $P(x, 0) = \sum_{k = 0}^\infty a_k \cdot x^k$
\subsubsection*{Konvergenz von Potenzr.}
\begin{enumerate}[label=\alph*., noitemsep]
    \item $P(x, x_0)$ konverg. in $c \Ra$ konverg. absolut $\forall x: |x - x_0| < |c - x_0|$
    \item Konverg. $P(x, x_0)$ in $c$ nicht abs. $\Ra$ divergiert $P(x, x_0) \forall |x - x_0| > |c - x_0|$
\end{enumerate}
\subsubsection*{Def. Konvergenzradius}
Sei $P(x, x_0)$ Potenzreihe. $\exists r \in \R_{\geq 0}$, dass $P(x, x_0)$ $\forall |x- x_0| < r$ konvergiert und $\forall |x - x_0| > r$ divergiert, dann ist $r$ der Konvergenzradius.
\subsubsection*{Konvergenzr. bestimmen}
$\bullet \lim_{n \ra \infty} \sqrt[n]{|a_n|} < 1$\\
$\bullet \lim_{n \ra \infty} |\frac{a_{n+1}}{a_n}| < 1$\\
Umformen:\\
$\bullet r = \lim_{n \ra \infty} \frac{1}{\sqrt[n]{|a_n|}}$
$\bullet r = \frac{|a_n|}{|a_{n+1}|}$
\subsection*{Exponentialreihe}
\subsubsection*{Definition}
$\exp(x) = \sum_{k = 0}^\infty \frac{x^k}{k!}$. $e := \exp(1)$ gilt
\subsubsection*{Konvergenz von Exp.Reihen.}
$\forall x \in \R: \exp(x)$ absolut konvergent.
\subsubsection*{Eigenschaften}
%\begin{enumerate}[label=\alph*., noitemsep]
    a) $\forall x, y \in \R: \exp(x + y) = \exp(x) \cdot \exp(y)$
    b) $\forall x \in \R: \exp(-x) = \frac{1}{\exp (x)}$
    c) $\forall x \in \R: \exp(x) > 0$
    d) $\forall n \in \Z: \exp(n) = e^{n}$ 
%\end{enumerate}
\subsection*{\texorpdfstring{$\exp$}{Exponentialfunktion} als Folgengrenzwert}
Gilt $\forall x \in \R$: $\exp(x) = \sum_{k = 0}^\infty \frac{x^k}{k!} = \lim\limits_{n \ra \infty} \left(1 + \frac{x}{n}\right)^n$. Für $x = 1$ gilt: $e = \sum_{k = 0}^\infty \frac{1}{k!} = \lim\limits_{n \ra \infty}\left(1 + \frac{1}{n}\right)^n$


\section*{Funktionen}
%\subsection*{Definition}
%$A, B$ nichtleere Mengen.Funktion $f$ ordnet jedem $x \in A$ eindeutig $y \in B$ zu. Schrift:$A \ra B$. Zugeordnetes Element auch als $f(x)$.\\
%$f:A \ra B$,
%$A$ Definitionsbereich,\\
%$B$ Bild-/Zielbereich\\
%$f(A) \subseteq B$ Bildmenge/Bild von $f$
\subsection*{Injektiv, ...}
%\begin{enumerate}[label={}, noitemsep, leftmargin=*]
 Injektiv: $x_1 \neq x_2 \Ra f(x_1) \neq f(x_2)$\\
 Surjektiv:$\forall y \in B \exists x \in A: f(x) = y$\\
 Bijektiv: Injektiv + Surjektiv
%\end{enumerate}

%%%\subsection*{Rechenregeln}
%Sei $f, g: A \ra \R$ Funktionen und $c \in \R$. Dann gilt:
%\begin{itemize}[noitemsep, leftmargin=*]
%%%     $\bullet (f + g)(x) := f(x) + g(x)$\\
%%%     $\bullet (cf)(x) := c f(x)$\\
%%%     $\bullet (f \cdot g)(x) := f(x) \cdot g(x)$\\
     %Sei $A':=\{x \in A| g(x) \neq 0\}$, dann Funktion $\frac{f}{g}: A' \ra \R$ definiert: 
%%%    $\bullet g(x) \neq 0$:$\left(\frac{f}{g}\right)(x) := \frac{f(x)}{g(x)}$\\
%%%    $\bullet f(A) \subseteq B \Ra (g \circ f)(x) := g(f(x))$
%\end{itemize}
\subsection*{Umekehrfunktion}
$f^{-1}: B \ra A$ Umkehrfunktion falls:\\
%\begin{itemize}[leftmargin=*, noitemsep]
    $(f^{-1} \circ f) (x) = f^{-1}(f(x)) = x, x \in A$\\
    $(f \circ f^{-1})(x) = f(f^{-1}(x)) = x, x \in B$
%\end{itemize}
\subsubsection*{Bijektiv-Umkehrfunktion}
$\exists f^{-1}$, $\eq$ $f$ bijektiv.
\subsubsection*{Monotonie Umkehrfunktion}
%$A \subseteq \R$, $f: A \ra B$ Funktion mit $B := f(A) \subseteq \R$. 
$f$ streng monoton $\Ra$ $f^{-1}:B \ra A$ existiert + streng mon. (im g. Sinne)
\subsection*{Beschränktheit}
$f: A \ra B$ heißt nach oben/unten Beschränkt, wenn Bildmenge $f(A)$ oben/unten beschränkt.
\subsection*{Monotonie}
Sei $A \subseteq \R$, $f: A \ra \R$, dann
%\begin{itemize}[leftmargin=*, noitemsep]
    $\bullet$ mon. wachsend: $f(x) \leq f(x')$
    $\bullet$ streng mon. wachs.: $f(x) < f(x')$
    $\bullet$ mon. fallend: $f(x) \geq f(x')$
    $\bullet$ streng mon. fall.: $f(x) > f(x')$
%\end{itemize}
$\forall x, x' \in A$ mit $x < x'$.
\subsection*{Berührpunkt}
$A \subseteq \R$, $a \in \R$, Dann ist $a$ Berührpunkt von $A$, falls $\forall \varepsilon \in \R, \varepsilon > 0 \ \exists b \in  (a - \varepsilon, a + \varepsilon): b \in A$
\subsection*{Grenzwerte Funktionen}
%Sei $f: A\in \R \ra \R$ und $a \in \R$ Berührpunkt von $A$. $\lim_{n \ra \infty} x_n = a \Ra \lim_{n \ra \infty} f(x_n) = c.$\\
%Analog definieren wir:
% $\lim_{x \ra \infty} f(x) = c$, wenn $A$ oben/unten unbeschränkt und $\forall (x_n)$ mit $\lim_{n \ra \infty} x_n = \pm\infty$ gilt $\lim_{n \ra \infty} f(x_n) = c$\\
 %\begin{enumerate}[label=\arabic*., noitemsep, leftmargin=*]
     1. Rechtssei.Grenzw.: 
     $\lim\limits_{n \searrow a} f(x) = x$\\
     %, wenn $a$ Berührpunkt von $A \cap (a, \infty)$ und $\forall$ $(x_n)$ mit $x_n \in A$, $x_n > a$ und $\lim_{n \ra \infty} x_n = a$ gilt: $\lim_{n \ra \infty} f(x_n) = c$
     2. Linkssei. Grenzw.: 
     $\lim\limits_{n \nearrow a} f(x) = x$
     %, wenn $a$ Berührpunkt von $A \cap (a, \infty)$ und $\forall$ $(x_n)$ mit $x_n \in A$, $x_n < a$ und $\lim_{n \ra \infty} x_n = a$ gilt: $\lim_{n \ra \infty} f(x_n) = c$
 %\end{enumerate}
\subsection*{Satz 4.20}
$\lim_{x \ra a} f(x) = f(a) $ $\eq$\\ $\lim_{x \nearrow a} f(x) = \lim_{x \searrow a} f(x) = f(a)$
\subsection*{Stetigkeit}
%Sei $f: A \ra \R$ Funktion,$a \in A$. 
$f$ stetig in $a$, falls $\lim_{x \ra a} f(x) = f(a)$. $f$ stetig, falls $\forall a \in A: f$ stetig
\subsubsection*{\texorpdfstring{$\varepsilon$}{Epsilon}-\texorpdfstring{$\delta$}{Delta}-Kriterium}
%Sei $A \subseteq \R$ und $f: A \ra \R$ funktion. $f$ ist g.d.w. in $a \in A$ stetig, wenn: 
$\forall \varepsilon > 0 \ \exists \delta > 0 \ \forall x \in A: |x- a| < \delta \Ra |f(x) - f(a)| \varepsilon$
%\subsubsection*{Operationen Stetigkeit}
%$f, g: A \ra \R$ in $a \in A$ stetig und $c \in \R$. Dann auch folgendes in $a$ stetig:
%\begin{enumerate}[label=\alph*., noitemsep]
%    $a) f + g: A \ra \R$
%    $b) c \cdot f: A \ra \R$\\
%    $c) f \cdot g: A \ra \R$\\
%    $d) \frac{f}{g}: A' \ra \R$, falls $g(a) \neq 0$\\
%    $e) g \circ f: A \ra \R$, falls $f$ in $a$ und $g$ in $f(a) = b$ stetig
%\end{enumerate}
\subsection*{Zwischenwertsatz}
Sei $f: [a, b] \ra \R$ stetig mit $f(a) \lessgtr 0 \lessgtr f(b)$ $\Ra$ $\exists x \in (a, b)$ mit $f(c)= 0$.\\
Allgemeiner: $\forall y \in\R$: Wenn $f(a) \lessgtr y \lessgtr f(b)$, dann $\exists d \in (a, b):$ $f(d) = y$
\subsection*{Umekehrfunk. stet. Funk.}
%Sei $I \subseteq\R$ Intervall und $f: I \ra \R$ stetig + streng monoton. Dann bildet $f$ $I$ bijektiv auf $f(I)$ ab und $f^{-1}: f(I)\ra \R$ ist stetig.
$f$ stetig + streng mono. $\eq$ $f^{-1}:f(I)\ra \R$ stetig
%\subsection*{Min,Max-kompakt. Interv.}
%Auf $[a, b]$ jede stetige Funktion $f: [a, b] \ra \R$ beschränkt und nimmt Min/Max an.
\subsection*{Gleichmäßige Stetigkeit}
$f: A \ra \R$ gleichmäßig stetig wenn: $\forall \varepsilon > 0 \ \exists \delta > 0 \ \forall x, y \in A: |x - y| < \delta \Ra |f(x) - f(y)| < \varepsilon$\\
$f: A \ra \R$ auf $[a, b] \in A$ stetig $\Ra$ dort auch gleichm. stetig.
\subsection*{Polynom}
Polynomfunktion: $p(x) = a_n x^n + ... + a_1 x + a_0$. $Grad(p) = \max(n)$, wo $a_n \neq 0$
\subsection*{Rationale Funktion}
$p, q$ Polynome, $q(x) \neq 0$, dann ist $r(x) = \left(\frac{p}{q}\right)(x) = \frac{p(x)}{q(x)}$ rat. Funk.
%\subsection*{Polynomdivision}
%\polyset{style=C, div=:,vars=x}
%\polylongdiv{x^2 - x + 1}{x-1}
\subsection*{Linearfaktoren}
$p(x)$ o. Rest d. $q(x) = x - x_1$ teilbar, g.d.w. $x_1 \in \R$ Nullstelle von $p(x)$.
\subsection*{Exponentialfunktion}
$\exp: \R \ra \R_{> 0}$: $\exp(x) = \sum_{k = 0}^{\infty} \frac{x^k}{k!}$
%%%\subsubsection*{Eigenschaften \texorpdfstring{$\exp$}{Exponential}-Funktion}
%\begin{enumerate}[label=\alph*., noitemsep]
%%%    $a) \exp(x + y) = \exp(x) \cdot \exp(y)$\\
%%%    $b) \exp(-x) = \frac{1}{\exp(x)}$\\
%%%    $c) \exp(x) > 0$\\
%%%    $d) \forall n \in \Z: \exp(n) = e^n$\\
%%%    $e) \exp(x) = \lim_{n \ra \infty} (1 + \frac{x}{n})^n$\\
%%%    $f)$ streng mon. wachsend + bijektiv\\
%%%    $g) \lim_{x \ra 0} \frac{\exp(x) - 1}{x} = 1$
%\end{enumerate}
\subsubsection*{1. Satz vom Wachstum}
$\forall n \in \N_{0}$ gilt:
%\begin{itemize}[noitemsep]
    $\bullet \lim_{x \ra \infty} \frac{\exp(x)}{x^n} = \infty$
    $\bullet \lim_{x \ra - \infty} \exp(x) x^n = 0$
%\end{itemize}
\subsubsection*{2. Satz vom Wachstum}
%$\forall n \in \N$ gilt: $\lim_{x \ra \infty} \frac{\ln(x)}{\sqrt[n]{x}} = 0$. 
$\ln(x)$ wächst schwächer als $\sqrt[n]{x}$
%\subsection*{Logarithmus}
%%Umkehrfunktion von $\exp(x)$ ist natürlicher Logarith. $\ln: \R_{>0} \ra \R$
%\subsubsection*{Eigenschaften \texorpdfstring{$\ln(x)$}{des Logarithmus}}
%\begin{enumerate}[label=\alph*., noitemsep]
%    \item $\ln(\exp(x)) = \exp(\ln(x)) = x$
%    \item $\ln(1) = 0$ und $\ln(e) = 1$
%    \item $\ln(x) \begin{cases}
%    < 0 &, x \in (0, 1)\\
%    = 0 &, x = 1\\
%    > 0 &, x > 1
%    \end{cases}$
%    \item $\ln(xy) = \ln(x) + \ln(y)$
%    \item $n \in \Z: \ln(x^n) = n \ln(x)$
%    \item $\ln(x)$ ist stetig
%\end{enumerate}
%%%\subsection*{allgemeine Exp.funktion}
%Sei $a \in \R_{> 0}$.$\exp_a: \R\ra \R: \exp_a(x) := \exp(x \ln(a))$. Schreiben auch $a^x$ statt $\exp_a(x)$.
%%%$\exp_a(x) := \exp(x \ln(a)) = a^x$
%%%\subsection*{Eigenschaft. allg. Potenzen}
%\begin{enumerate}[label=\alph*., noitemsep]
%%%    $a) a^x = \exp_a(x)$ stetig $\forall x \in \R$\\
%%%    $b) \forall n \in \Z$: $\exp_a(x) = a^n$\\
%%%    $c) a^{x+y} = a^x a^y$\\
%%%    $d) (a^x)^y = a^{xy}$\\
%%%    $e) a^x b^x = (ab)^x$\\
%%%    $f) \forall p \in \Z, q \in \N\backslash\{1\}: a^{\frac{p}{q}} = \sqrt[q]{a^p}$
%\end{enumerate}
%%%\subsection*{Log zu allg. Basen}
%%%Sei $a \in \R_{> 0}\backslash\{1\}$, dann $\log_a: \R_{> 0} \ra \R:$ $\log_a(x) := \frac{\ln(x)}{\ln(a)}$
\subsection*{Funktionssymmetrie}
%\begin{itemize}[noitemsep]
    $\bullet$ achsen(gerade): $f(-x) = f(x)$\\
    $\bullet$ punkt(ungerade):$f(-x) = -f(x)$
%\end{itemize}
%%%\subsection*{Hyperbolische Funktionen}
%\begin{itemize}[noitemsep]
%%%    $\bullet \cosh(x) := \frac{e^x + e^{-x}}{2}$\\
%%%    $\bullet \sinh(x) := \frac{e^x - e^{-x}}{2}$\\
%%%    $\bullet \tanh(x) := \frac{\sinh(x)}{\cosh(x)} = \frac{e^x - e^{-x}}{e^{-x}  + e^x}$
%\end{itemize}
%%%\subsubsection*{Eigensch. hyperb. Funkt}
%\begin{enumerate}[label=\alph*., noitemsep]
%%%    $a) \exp(x) = \cosh(x) + \sinh(x)$\\
%%%    $b) \cosh^2(x) - \sinh^2(x) = 1$\\
%%%    $c) \cosh(x) = \sum_{k = 0}^{\infty} \frac{x^{2k}}{(2k)!}$\\
%%%    $d) \sinh(x) = \sum_{k = 0}^{\infty} \frac{x^{2k+1}}{(2k + 1)!}$\\
%%%    $e) \cosh(x + y) = \cosh(x)\cosh(y) + \sinh(x) \sinh(y)$\\
%%%    $f) \sinh(x + y) = \sinh(x) \cosh(y) + \sinh(x) \cosh(y)$
%\end{enumerate}
\subsection*{komplexe \texorpdfstring{$\exp$}{Exponential}-Funktion}
$\C \ra \C$, $\exp(z) = e^z = \sum_{k = 0}^\infty \frac{z^k}{k!}$
%%%\subsection*{Trigonom. Funktionen}
%$\sin/\cos: \R \ra \R$, 
%%%$\tan: \{x| \cos(x) \neq 0\} \ra \R$\\
%\begin{itemize}[noitemsep, leftmargin=*]
    %%%$\bullet \cos(x) := \text{Re}(e^{ix}) = \frac{e^{ix} + e^{-ix}}{2}$\\
    %%%$\bullet \sin(x) := \text{Im}(e^{ix}) = \frac{e^{ix} - e^{-ix}}{2}$\\
    %%%$\bullet \tan(x) := \frac{\sin(x)}{\cos(x)} = \frac{ie^{-ix} - ie^{ix}}{e^{-ix} + e^{ix}}$
%\end{itemize}
%%%\subsubsection*{Eigenschaften trig. Funkt.}
%\begin{enumerate}[label=\alph*., noitemsep]
    %%%$a) \exp(ix) = \cos(x) + i \sin(x)$\\
    %%%$b) \cos^2(x) + \sin^2(x) = 1$\\
    %%%$c) |\sin(x)| \leq 1$ und $|\cos(x)| \leq 1$\\
    %%%$d) \cos(x) = \sum_{k = 0}^\infty (-1)^k \frac{x^{2k}}{(2k)!}$\\
    %%%$e) \sin(x) = \sum_{k = 0}^{\infty} (-1)^k \frac{x^{2k +1}}{(2k + 1)!}$\\
    %%%$f) \cos(x + y) = \cos(x) \cos(y) - \sin(x) \sin(y)$\\
    %%%$g) \sin(x + y) =  \sin(x) \cos(y) + \cos(x)\sin(y)$
%\end{enumerate}
%%%\subsubsection*{Abschätzung Sin-Cos}
%%%Für $x \in (0, 2]$ gilt:\\
%%%%\begin{itemize}[leftmargin=*, noitemsep]
%%%    $\bullet 1 - \frac{x^2}{2} < \cos(x) < 1 - \frac{x^2}{2} + \frac{x^4}{4!}$\\
%%%    $\bullet x - \frac{x^3}{3!} < \sin(x) < x$
%%%%\end{itemize}
%%%\subsubsection*{Folgerung Def. Pi}
%%%\renewcommand{\arraystretch}{1.5}
%%%\begin{tabular}{m{1.1em}m{0.2em}m{0.2em}m{0.2em}m{0.2em}m{0.2em}m{0.2em}m{0.2em}m{0.2em}}
%%%    $x$ & $0$& $\frac{\pi}{6}$ & $\frac{\pi}{4}$ & $\frac{\pi}{3}$ & $\frac{\pi}{2}$ & $\pi$ & $\frac{3\pi}{2}$ & $2\pi$ \\\hline
%%%    $\cos $ & $1$ & $\frac{\sqrt{3}}{2}$       & $\frac{1}{\sqrt{2}}$ & $\frac{1}{2}$        & $0$  & $-1$ & $0$  & $1$ \\\hline
%%%    $\sin $ & $0$ & $\frac{1}{2}$              & $\frac{1}{\sqrt{2}}$ & $\frac{\sqrt{3}}{2}$ & $1$  & $0$  & $-1$ & $0$ \\\hline
%%%    $\tan $ & $0$ & $\frac{1}{\sqrt{3}}$       & $1                 $ & $\sqrt{3}$           & $-$    & $0$   & $-$   & $0$\\
%%%\end{tabular}
%%%%Allgemein gilt für alle $x \in \R$:
%%%$\cos(x + \pi/2) = \sin(x + \pi) =  - \sin(x)$\\
%%%$\cos(x + 2 \pi) = \sin(x + \pi/2) = \cos(x)$
%%%$\cos(x + \pi) = -\cos(x)$\\
%%%$\sin(x + 2\pi) = \sin(x)$
%%%\renewcommand{\arraystretch}{0.5}
%hier ist zu überlegen, ob dies vielleicht weg kann
\subsection*{Periodische Funktionen}
$f: \R \ra \R$ periodische Funktion, wenn $\exists p > 0$, sodass $f(x) = f(x + p), \forall x \in \R$.\\
$\min(p) \in \R_{>0}$ heißt Periode.
%\subsection*{Polarkoordinaten \texorpdfstring{$\C$}{komplexer Zahlen}}
%$\forall z \in \C \ \exists \phi \in \R$, sodass $z = |z|e^{i \phi} = |z| \cos(\phi) + i |z| \sin(\phi)$. Für $z \neq 0$ ist $\phi$ bis auf eine Addition mit Vielfachen von $2 \phi$ eindeutig. Das Paar $(|z|, \phi)$ bezeichnet wir als Polarkoordinaten von $z$ und $\phi$ als Argument von $z$.
\section*{Differentialrechnung}
\subsubsection*{Definition}
Sei $a \in A \subseteq \R$ und $f: A \ra \R$. $f$ heißt in $a$ diffbar, falls Grenzwert $\lim_{\underset{x \in A\backslash\{a\}}{x \ra a}} \frac{f(x) - f(a)}{x - a}$ exist. 
Oder:\\$f'(x) = \lim_{h \ra 0} \frac{f(a + h) - f(a)}{h}$.\\
$\forall a \in A$ Grenzwert exist. $\Ra$ $f$ diffbar.

\subsubsection*{Diffbar \texorpdfstring{$\Ra$}{folgt} Stetig}
%$f:A \ra \R$ in $a \in A $ differenzierbar $\Ra$ in $a$ stetig
\subsection*{Satz 5.5}
%$f:[a, b] \ra \R$ diffbar für $c\in (a, b)$ g.d.w. links- + rechtsseitiger Grenzwert existieren und gleich sind. 
$f'_{-}(c) = \lim\limits_{x \nearrow c} \frac{f(x) - f(c)}{x - c} = f'_{+}(c) = \lim\limits_{x \searrow c} \frac{f(x) - f(c)}{x - c} \Ra$ $f'(c) = f'_{-}(c) = f'_{+}(c)$
\subsection*{Ableitungsregeln}
%$f, g: A \ra \R$ in $a \in A$ diffbar.
%\begin{enumerate}[label=\alph*., noitemsep]
    $a)$ Linearität:\\
    $(c \cdot f)'(a) = c \cdot f'(a)$
    $(f + g)'(a) = f'(a) + g'(a)$\\
    $b)$ Produktregel:\\
    $(f \cdot g)'(a) = f'(a) \cdot g(a) + f(a) \cdot g'(a)$\\
    $c)$ Quotientenr.: $g(x) \neq 0, \forall x \in A$\\
    $\left(\frac{f}{g}\right)'(a) = \frac{f'(a) \cdot g(a) - f(a) \cdot g'(a)}{g^2(a)}$\\
    $d)$ Kettenregel:\\
    $(g \circ f)'(a) = g'(f(a)) \cdot f'(a)$
%\end{enumerate}
\subsection*{Ableitung Umkehrf.}
%Sei $f:I\ra \R$ stetig streng monoton und $g = f^{-1}:J\ra\R$ mit $J = f(I)$. $f$ in $a \in I$ diffbar und $f'(a)\neq 0 \Ra g$ in $b = f(a)$ diffbar und es gilt: $g'(b) = \frac{1}{f'(a)} = \frac{1}{f'(g(b))}$
$f$ stetig,streng mon,diffbar, $g=f^{-1}$ $\Ra$ $g'(f(a)) = \frac{1}{f'(a)}$
\subsection*{Ableitung höherer Ord.}
$f^{(k + 1)}(a) := \left(f^{(k)}(a)\right)'$
%, falls $f^{(k)}(a)$ in $a \in A$ existiert. 
$f$ $k$-mal (stetig) diffbar oder $f$ $C^k$-stetig
\subsection*{Lokale Extrema}
%$f:(a, b) \ra \R$ in $x \in  (a, b)$ lok. Min/Max $f(x)$, wenn
$\exists\varepsilon > 0 \ \forall y, |x - y| < \varepsilon: f(a) \leq\geq f(y)$
\subsubsection*{Notw. Bed. lok. Extrema}
$f:(a, b) \ra \R: f'(x) = 0$
\subsubsection*{Hin. Bed. lok. Extrema}
$f:(a, b) \ra \R$ 2 mal diffbar in $x \in (a, b)$. $f''(x) > 0$: Minimum, $f''(x) < 0$: Maximum
\subsection*{Monoton. und Ableitung}
$f:[a,b]\ra \R$ stetig, diffbar in $(a, b)$:\\
%\begin{enumerate}[label=\alph*., noitemsep]
    $\bullet f'(x) \geq 0 \eq f$ mon. wachs.\\
    $\bullet f'(x) > 0 \Ra f$ streng mon. wach.\\
    $\bullet f'(x) \leq 0 \eq f$ mon. fall.\\
    $\bullet f'(x) < 0 \Ra f$ streng mon. fall.
%\end{enumerate}
\subsection*{Satz von Rolle}
Sei $a < b$ und $f:[a, b] \ra \R$ stetig (auf $(a, b)$ diffbar) mit $f(a) = f(b)$, dann: $\exists c \in (a, b): f'(c) = 0$
\subsection*{1. Mittelwertsatz}
Sei $a < b$ und $f:[a, b] \ra \R$ stetig (auf $(a, b)$ diffbar) mit $f(a) = f(b)$, dann: $\exists c \in (a, b): \frac{f(b) - f(a)}{b - a} = f'(c)$
\subsection*{2. Mittelwertsatz}
$f, g: [a, b]\ra \R$ stetig in $[a, b]$, diffbar in $(a, b)$. Sei 
%$\forall x \in (a, b):$ 
$g'(x)\neq 0$ $\Ra$ $g(a) \neq g(b)$ + $\exists c \in (a, b)$ mit $\frac{f(b) - f(a)}{g(b) - g(a)} = \frac{f'(c)}{g'(c)}$
\subsection*{Satz 5.21}
$f:(a, b) \ra \R$ 
%diffbar und im Punkt $x \in (a, b)$ 
$n +1$-mal diffbar. Falls $f'(x) = f^{(2)}(x) = ... = f^{(n)}(x) = 0$ und $f^{(n+1)}(x) \neq 0$, dann $f$ in $x$\\
%\begin{itemize}[leftmargin=*, noitemsep]
    $\bullet$ streng. lok. Min., falls $n$ ungerade und $f^{(n+1)}(x) > 0$\\
    $\bullet$ streng. lok. Max., falls $n$ ungerade und $f^{(n+1)}< 0$\\
    $\bullet$ kein Extremum, falls $n$ gerade
%\end{itemize}
\subsection*{Konvexität}
%$f:(a, b) \ra \R$ heißt konvex, wenn 
$\forall x_1, x_2 \in (a,b)\forall \lambda\in(0,1):$ $f(\lambda x_1 + (1 - \lambda) x_2) \leq \lambda f(x_1) + (1 - \lambda)f(x_2)$. Konkav wenn $-f$ konvex.
\subsubsection*{Satz 5.23}
$f$ konvex $\eq$ $\forall x \in (a, b): f''(x) \geq 0$
\subsection*{Wendepunkt}
$f$ in $x \in (a, b)$ Wendepunkt, wenn $\exists(\alpha, x), (x, \beta)$ für die gilt:\\
%\begin{enumerate}[label=\arabic*, noitemsep]
    $1.$ $f$ in $(\alpha, x)$ streng konvex + in $(x, \beta)$ streng konkav\\
    $2.$ $f$ in $(\alpha, x)$ streng konvex + in $(x, \beta)$ streng konvex
%\end{enumerate}
\subsubsection*{Satz 5.26}
%$f:(a, b) \ra \R$ 3 mal diffbar: $f'$ lok. Extremum in $x$ $\Ra$ Wendepunkt
%\begin{itemize}[noitemsep]
    $\bullet$ Not. Bed.: $f''(x) = 0$\\
    $\bullet$ Hin. Bed.: $f''(x) = 0$ $\land$ $f'''(x) \neq 0$
%\end{itemize}
\subsection*{Uneigentlicher Grenzwert}
$f:A \ra \R$ und $a$ Häufungspunkt. Falls $\forall K \in \R \ \exists \delta > 0$, sodass $f(x) > K$ für $|x - a| < \delta$, $\lim_{x \ra a}f(x) = \infty$.
\subsection*{L'Hospital}
%$f, g$ diffbar und $g(x) \neq 0$ und $g'(x) \neq 0$ für alle $x \in (a, b)$. Und 
a oder b gilt:\\
%\begin{enumerate}[label=\alph*., noitemsep]
    $a) \lim\limits_{x \searrow a} f(x) = \lim\limits_{x \searrow a} g(x)= 0$\\
    $b) \lim\limits_{x \searrow a} |f(x)| = \lim\limits_{x \searrow a} |g(x)| = \infty$\\
%\end{enumerate}
Dann:$\lim_{x\searrow a} \frac{f(x)}{g(x)} = \lim_{x \searrow a} \frac{f'(x)}{g'(x)}$. Analog für $\lim_{x \nearrow b}$ und $a, b = \pm \infty$.
\subsection*{Satz von Taylor}
\subsubsection*{Tylorsche Formel}
$f$
%$: A \ra \R$ 
($n + 1$)-mal stetig diffbar.\\
%$T_n, R_n: A \ra \R$ für beliebiges $n \in \N$ und $x_0\in A$ definiert als
$T_n(x) := \sum_{k = 0}^n \frac{f^{(k)}(x_0)}{k!}(x - x_0)^k$\\ %und 
$R_n(x) := f(x) - T_n(x)$\\%. Dann 
%$\exists y\in [x, x_0] \forall x \in A \backslash\{x_0\}$, sodass 
$R_n(x) = \frac{f^{(n+1)}(y)}{(n+1)!}(x - x_0)^{n+1}$
\subsubsection*{Taylorreihe/-polynom}
$f$
%$:A \ra \R$ 
beliebig oft diffbar\\
%in $x_0 \in A$. Dann heißt 
$T[f, x_0](x) = \sum\limits_{k = 0}^\infty \frac{f^{(k)}(x_0)}{k!} (x - x_0)^k$ 
%Taylorreihe von $f$ im Punkt $x_0$.\\
%Die $n$-te Teilsumme der Taylorreihe 
%$T_n[f, x_=](x)$ heißt Taylorpolynom vom Grad $n$ mit Entwick.punkt $x_0$:
$T_n[f, x_0](x) = \sum\limits_{k = 0}^n \frac{f^{(k)}(x_0)}{k!} (x - x_0)^k$
\section*{Integralrechnung}
%\subsection*{Zerlegung}
%$[a, b]\subset \R$ und endliche Anzahl Punkte $x_0, ..., x_n$ mit $a = x_0 < ... < x_n = b$. Dann heißt $Z = (x_0, ..., x_n)$ Zerlegung von $[a, b]$ und $|Z| := \max \{x_i - x_{i - 1} | i = 1, ..., n\}$ ist Feinheitsmaß von $Z$. Zerlegung heißt äquidistant, wenn die Intervalle $[x_{i - 1}, x_i]$ für $i = 1, ..., n$ alle gleich groß sind.
%\subsection*{Treppenfunktion}
%$Z = (x_0, ..., x_n)$ Zerlegung von $[a, b]$, dann heißt $\phi:[a, b]\ra \R$ Treppenfunktion, wenn sie auf jedem Teilintervall $(x_{k-1}, x_k)$ konstant ist. $\mathcal{T}[a,b]$ Menge aller Treppenfunktionen auf $[a, b]$
\subsubsection*{Integral für Trep.}
$\phi \in \mathcal{T}[a,b]$ Treppenfunktion bezüglich $Z = (a = x_0, ..., x_n = b)$ und seien $\phi(x) = c_k$ konstante Funktionsabschnitte von $\phi$ für $x \in (x_{k-1}, x_k)$. Def.: Integral von $\phi$ auf $[a, b]$ als: $\int_{a}^b \phi(x) dx := \sum_{k = 1}^n c_k (x_k - x_{k - 1})$.
\subsubsection*{Monoton. Treppen.}
$\phi, \psi \in \mathcal{T}[a, b]$ $\forall x \in [a, b]: \phi(x) \leq \psi(x) \Ra \int_{a}^b \phi(x) dx \leq \int_a^b \psi(x) dx$
\subsubsection*{Einschließen Treppen.}
$f:[a, b] \ra \R$ integrierbar $\eq$ $\forall \varepsilon > 0$ $\bar S(Z, f)$ und $\underline{S}(Z, f)$ existieren mit $\bar S (Z, f) - \underline{S}(Z, f) \leq \varepsilon$
\subsection*{Ober-/Untersumme}
Obersumme: $\bar S(Z, f) := \int_a^b \bar \phi(x) dx = \sum_{k = 1}^n \bar c_k (x_k - x_{k - 1})$\\
Untersumme: $\underline{S}(Z, f) := \int_a^b \underline{\phi}(x) dx = \sum_{k = 1}^n \underline{c}_k (x_k - x_{k -1})$
\subsection*{Ober-/Unterintegral}
Oberint: $\overline{\int_a^b} f(x) dx := \inf \{\bar S(Z, f)\}$\\
Unterint: $\underline{\int_a^b}f(x) dx := \sup \{\underline{S}(Z, f)\}$
\subsection*{Riemann-Integral}
$f$ ist integrierbar, wenn $\underline{\int_{a}^b} f(x) dx = \overline{\int_a^b} f(x) dx$. Integral von $f$ ist dann $\int_a^b f(x) dx := \underline{\int_a^b} f(x) dx$
\subsection*{Stetig \texorpdfstring{$\Ra$}{folgt} intbar}
$f$ auf $[a, b]$ stetig $\Ra$ $f$ auf $[a, b]$ intbar.
\subsection*{Monoton \texorpdfstring{$\Ra$}{folgt} intbar}
$f$ monoton auf $[a, b]$ $\Ra$ $f$ auf $[a, b]$ intbar
\subsection*{Verfeinerung}
$Z, Z'$ Zerlegungen.
Verfeinerung: $\tilde Z$ enhtält alle Elemente von $Z$ auf gleichem Intervall\\
Überlagerung: $\hat Z = Z + Z'$
\subsubsection*{Zerlegungswechsel}
$f$ auf $[a, b]$ beschränkt mit $|f(x)| \leq K$ und $Z$ Zerlegung von $[a, b]$ mit Feinheitsmaß $|Z|$. Zerlegung $\tilde Z$ entstehe aus $Z$ durch Hinzunahme eines zusätz. Punkts. Dann gilt:\\
%\begin{enumerate}[label=\alph*., noitemsep]
    $\underbar S(Z, f) \leq \underbar S(\tilde Z, f) \leq \underbar S(Z, f) + 2K |Z|$\\
    $\bar S(Z, f) \geq \bar S(\tilde Z, f) \geq \bar S(Z, f) - 2K |Z|$
%\end{enumerate}
\subsection*{Integralwertbestimmung}
$f:[a, b]$ beschränkt und $(Z_n)$ Folge von Zerlegungen mit $\lim_{n \ra \infty} |Z_n| = 0$. Dann gilt:\\
%\begin{enumerate}[label=\alph*., noitemsep]
    $a) \lim_{n \ra \infty} \underbar S(Z_n, f) = \underline{ \int_a^b} f(x) dx$\\
    $b) \lim_{n \ra \infty} \bar S(Z_n, f) = \overline{\int_a^b} f(x) dx$
%\end{enumerate}
\subsection*{Rieman. Zwischensumme}
$f:[a, b] \ra \R$ integrierbar und $(Z_n)$ Zerlegungsfolge mit $\lim_{n \ra \infty} |Z_n| = 0$. $\forall Z_n = (x_0, ..., x_m)$ sei $\phi_n \in \mathcal{T}[a,b]$ Treppenfunktion mit $\phi_n(x) := f(\zeta_k)$ für $x \in [x_{k - 1}, x_k)$ mit beliebigen $\zeta_k \in [x_{k - 1}, x_k]$ und $\phi_n(b) = f(b)$. Dann konvergiert $(S_n)$ der Riemannschen Zwischensummen $S_n := \int_a^b \phi_n(x) dx = \sum_{k = 1}^n f(\zeta_k) (x_k - x_{k-1})$ gegen Integral: $\int_{a}^b f(x) dx = \lim_{n \ra \infty} S_n$.
\subsection*{Integraleigeschaften}
$f, g$ auf $[a, b]$ intbar und $\lambda, \mu \in \R$\\
%\begin{enumerate}[label=\alph*., noitemsep]
    $a) $ Linearität $\int_a^b \lambda f(x) + \mu g(x) dx = \lambda\int_a^b f(x) dx + \mu \int_a^b g(x) dx$\\
    $b) $ Monotonie:$\int_a^b f(x) dx \leq \int_a^b g(x) dx$\\
    $c) $ Besch.:$\left|\int_a^b f(x) dx\right| \leq \int_a^b |f(x)| dx$\\
    $d) $ \small$\int_a^b f(x) dx = \int_a^c f(x) dx + \int_c^b f(x) dx$\\
    $e) $ $\int_b^a f(x) dx := - \int_a^b f(x) dx$\\
    $f) $ $\int_c^c f(x) dx := 0$
%\end{enumerate}
\subsection*{Mittelwertsatz Integralr.}
$f:[a, b]\ra \R$ intbar und $\forall x \in [a, b]: m \leq f(x) \leq M$. Dann gilt: $m(b - a) \leq \int_a^b f(x) dx \leq M(b - a)$. \\
Ist $f$ auch stetig, dann $\exists c \in (a, b)$ mit $\int_a^b f(x) dx = (b - a)f(c)$
\subsection*{Hauptsatz D./I.Rechnnung}
$f: I \ra \R$ stetig und $c \in I$. Sei $F: I \Ra \R$ für $x \in I$ definiert als $F(x) = \int_c^x f(t) dt$. Dann folgt:\\
%\begin{enumerate}[label=\alph*., noitemsep]
    $a)$ $F$ stetig diffbar $\land$ $F'(x) = f(x)$\\
    $b)$ Für beliebige $a, b \in I$ gilt $\int_a^b f(t) dt = F(b) - F(a)$
%\end{enumerate}
\subsection*{Stammfunk., unb. Int.}
$F(x) = \int f(x) dx$. Es gilt $[F(x)]_a^b = F(b) - F(a)$
\subsection*{Bekannte Integrale}
\begin{adjustwidth}{-1.2em}{}
\begin{tabular}[leftmargin=*]{p{2em}p{4.6em}p{10em}}
     $f(x)$ & $F(x)$ & Def.Bereich $f$ \\
     $c$ & $cx$ & $\R \text{ für } c\in \R$ \\
     $x^n$ & $\frac{1}{n+1} x^{n +1}$ & $\R \text{ für } n \in \R$ \\
     $\frac{1}{x^n}$ & $- \frac{1}{n-1} \frac{1}{x^{n-1}}$ & $\R \backslash \{0\}$,$n \in \N\backslash\{1\}$ \\
     $x^\alpha$ & $\frac{1}{\alpha + 1} x^{\alpha + 1}$ & $\R_{>0}, \alpha \in \R\backslash\{-1\}$ \\
     $\frac{1}{x}$ & $\ln(|x|)$ & $\R\backslash\{0\}$ \\
     $e^x$ & $e^x$ & $\R$ \\
     $a^x$ & $\frac{a^x}{\ln(a)}$ & $\R, a > 0, a \neq 1$ \\
     $\ln(|x|)$ & $x \cdot \ln(|x|) - x$ & $\R\backslash\{0\}$ \\
     $\frac{1}{1 + x^2}$ & $\arctan(x)$ & $\R$ \\
     $\frac{1}{1 - x^2}$ & $\frac{1}{2}\ln(|\frac{x + 1}{x - 1}|)$ & $\R\backslash\{-1, 1\}$ \\
     $\frac{1}{\sqrt{1 + x^2}}$ & $arsinh(x)$ & $\R$ \\
     $\frac{1}{\sqrt{1 - x^2}}$ & $\arcsin(x)$ & $(-1, 1)$ \\
     $\sin(x)$ & $-\cos(x)$ & $\R$ \\
     $\cos(x)$ & $\sin(x)$ & $\R$ \\
     $\tan(x)$ & $- \ln (|\cos(x)|)$ & $\R\backslash\{x = \frac{\pi}{2} + k \cdot \pi\}$ \\
\end{tabular}
\end{adjustwidth}
\subsection*{Partielle Integration}
$f, g:[a, b]\ra \R$ stetig diffbar.:\\
$\int_a^b f(x) g'(x) dx = [f(x) g(x)]_a^b - \int_a ^b f'(x) g(x) dx$\\
Unbestimmte Form: \\
$\int f(x) g'(x) dx = f(x) g(x) - \int f'(x) g(x) dx$
\subsection*{Integration d. Substitution}
$f$ stetig, $g$ stetig diffbar:\\
$\int_a ^b f(g(t)) g'(t) dt = \int_{g(a)}^{g(b)} f(x) dx$\\
Unbestimmte Form:\\
$\int f(g(t)) g'(t) dt = F(g(t))$
\subsection*{Integralvereinfachung}
%\begin{enumerate}[label=\alph*., noitemsep]
    $a) \int f(ax + b) dx = l\frac{1}{a} F(ax + b)$\\
    $b) \int f(x) f'(x) dx = \frac{1}{2} f^2(x)$\\
    $c) \int \frac{f'(x)}{f(x)} dx = \ln (|f(x)|)$
%\end{enumerate}
\subsection*{Integrale ü. uneig. Int.}
$\int_a^\infty f(x) dx := \lim_{c \ra \infty} \int_a^c f(x) dx$\\
$\int_{- \infty}^a f(x) dx := \lim_{c \ra - \infty} \int_c^a f(x) dx$\\
$\int_{-\infty}^\infty f(x) dx := \int_{-\infty}^c f(x) dx + \int_c^\infty f(x) dx$
\subsection*{Int. ü. offene Intervalle}
$f:(a, b) \ra \R$. $\forall I = [\alpha, \beta]\subset(a, b):$ integrierbar. Sei $c \in (a, b)$ beliebig:\\
%\begin{itemize}[noitemsep,label={}, leftmargin=*]
    $\int_c^b f(x) dx := \lim_{\beta \nearrow b} \int_c^\beta f(x) dx$\\
    $\int_a^c f(x) dx := \lim_{\alpha \searrow a} \int_\alpha^c f(x) dx$\\
    $\int_a^b f(x) dx := \int_a^c f(x) dx + \int_c^b f(x) dx$\\
%\end{itemize}

\subsection*{Bekannte Reihen}
\subsubsection*{Harmonische Reihe}
$\sum_{k=1}^\infty \frac{1}{k}$, divergent
\subsubsection*{Geometrische Reihe}
$\sum_{k = 0}^\infty q^k$, Grenzwert: $\frac{1}{1-q}$, konvergiert für $|q|< 1$, divergiert für $|q| \geq 1$
\subsection*{3. Binom. Erweiterung}
$(b - \sqrt{a}) = (b - \sqrt{a}) \frac{(b + \sqrt{a})}{(b + \sqrt{a})}$

\subsection*{Partialbruchzerlegung}
1. Polynomdivision (falls nötig)\\
2. Nullstelllen bestimmen (Umformen)\\
3. Aus Nullstellen: $\frac{A}{n_1} + ... + \frac{Z}{n_k}$\\
4. Auf einen Nenner bringen\\
5. Umformen in ähnliche Form wie Ausgangsfunktion\\
6. Aus den Koeffizienten A-Z Gleichungssysteme lösen
\end{multicols*}
\end{document}

